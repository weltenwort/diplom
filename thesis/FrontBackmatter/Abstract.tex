%*******************************************************
% Abstract
%*******************************************************
%\renewcommand{\abstractname}{Abstract}
\pdfbookmark[1]{Abstract}{Abstract}
\begingroup
\let\clearpage\relax
\let\cleardoublepage\relax
\let\cleardoublepage\relax

\chapter*{Abstract}
The steady increase in quantity of digital images creates demand for more
efficient and accurate means to search and categorize visual information in
large databases. This thesis describes the Fast Discrete Curvelet Transform
(FDCT) in the context of sketch-based image retrieval. The properties of the
FDCT suggest, that it might be well suited to represent features found in
hand-drawn sketches. After examining the general structure of image retrieval
systems, several variations of processing pipelines are implemented, that use
sketches as query input to retrieve similar images from a database using global
and local descriptors. The quality of the retrieved results is evaluated using
two benchmarks, that cover both cross-domain and intra-domain scenarios.
The results show, that retrieval methods based on the FDCT can compete with
previously published algorithms, even for cross-domain queries. It is also
shown that the choice between descriptors based on global image characteristics
or local bag-of-features descriptors depends on the nature of the images used.

\vfill

\pdfbookmark[1]{Zusammenfassung}{Zusammenfassung}
\chapter*{Zusammenfassung}
Der stete Zuwachs an digitalen Bildern führt zu einem zunehmenden Bedarf an
effizienten Methoden, um große Bilddatenbanken zu durchsuchen und deren Inhalt
zu kategoriesieren. Diese Diplomarbeit beschreibt die Fast Discrete Curvelet
Transform (FDCT) im Zusammenhang mit Bildersuche anhand von Handzeichnungen.
Die Eigenschaften der FDCT legen nahe, dass sie gut geeignet sein könnte um
Features in Handzeichnungen abzubilden. Nach der Untersuchung der
grundsätzlichen Struktur von Systemen zur Bildersuche werden mehrere
Variationen eines solchen Systems implementiert, das Handzeichnungen zur Suche
mittels globaler und lokaler Deskriptoren verwendet. Die Qualität der
Suchergebnisse wird anhand von zwei Benchmarks gemessen, die sowohl
domänenübergreifende als auch domäneninterne Abfragen abdecken.
Die Ergebnisse zeigen, dass Suchmethoden auf Basis der FDCT auf gleichem Niveau
wie zuvor veröffentlichte Algorithmen liegen. Sie zeigen außerdem, dass die
Wahl zwischen Deskriptoren, die auf global Bildeigenschaften basieren, und
solchen, die lokale bag-of-features Ansätze verfolgen, stark von der Art des
verwendeten Bildmaterials abhängig ist.

\endgroup			

\vfill
