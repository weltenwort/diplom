% vim: set fdm=marker:
\pdfminorversion=4
\documentclass{beamer}

% theme {{{
%\usetheme{Antibes}
\usetheme[compress]{Dresden}
%\usecolortheme{dolphin}
\usecolortheme{rose}
%\usefonttheme{serif}
%}}}

% packages {{{
\usepackage[english]{babel}
\usepackage[utf8]{inputenc}
\usepackage[T1]{fontenc}
%}}}

% title {{{
\title{Analysis of Image Tranforms for Sketch-based Retrieval}
\subtitle{Diploma Thesis}
\author{Felix Stürmer}
\institute[Fakultät IV - TU Berlin]
{
    Technische Universität Berlin\\
    Fakultät IV - Elektrotechnik und Informatik\\
    Computer Graphics
}
\date{02.11.2012}
\subject{Computer Graphics}
%}}}

\begin{document}
% document {{{

% titlepage {{{
\begin{frame}
  \titlepage
\end{frame}
%}}}

% toc {{{
\begin{frame}{Outline}
  \tableofcontents
  % You might wish to add the option [pausesections]
\end{frame}
%}}}

% introduction and background {{{
\section{Introduction and Background}
\subsection{Motivation}
\begin{frame}{Motivation}
    foo
\end{frame}

\subsection{Challenges of CBIR}
\begin{frame}{Challenges of CBIR}
    foo
\end{frame}

\subsection{Anatomy of a CBIR System}
\begin{frame}{Anatomy of a CBIR System}
    foo
\end{frame}
% }}}

% solution {{{
\section{Proposed Solution}
\subsection{Acquisition}
\begin{frame}{Acquisition}
    foo
\end{frame}

\subsection{The Curvelet Transform}
\begin{frame}{The Curvelet Transform}
    foo
\end{frame}

\begin{frame}{The Fast Discrete Curvelet Transform}
    foo
\end{frame}

\subsection{Feature Extraction}
\begin{frame}{Global Feature Extraction}
    foo
\end{frame}

\begin{frame}{Local Feature Extraction}
    foo
\end{frame}

\subsection{Ranking}
\begin{frame}{Ranking}
    foo
\end{frame}
%}}}

% results {{{
\section{Results}
\subsection{Benchmarking}
\begin{frame}{Benchmarking Method}
    foo
\end{frame}

\subsection{Cross-Domain Results}
\begin{frame}{Cross-Domain Results}
    foo
\end{frame}

\subsection{Intra-Domain Results}
\begin{frame}{Intra-Domain Results}
    foo
\end{frame}
%}}}

% conclusion {{{
\section{Conclusions}
\begin{frame}{Conclusions}
    foo
\end{frame}
%}}}

%}}}
\end{document}
