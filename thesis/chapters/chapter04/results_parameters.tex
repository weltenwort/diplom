\subsection{Parameter Variations}

In an attempt to improve the results, the best performing configurations
presented in the previous sections will be re-used with varying parameter
values (Table~\ref{tab:results_best_performers}). The following sections
compare the results of changing the parameters $N_{\theta}$, $P$, $G$ and
$\sigma$.

\begin{table}[h]
    \centering
    \pgfplotstableread[]{results/best_performers.csv}\resultsbestperformers
    \plottablexbars{imagereader,features,gridsize,patchsize,cannysigma,metric}{\resultsbestperformers}
    \caption[Best Performing Configurations]{
        Best Performing Configurations with default assumptions $N_s=4$ and
        $N_{\theta}=12$.
    }
    \label{tab:results_best_performers}
\end{table}

\FloatBarrier
\subsubsection{Curvelet Angles}

The parameter $N_{\theta}$ controls the number of angles, the curvelet
coronization is divided into (see Section~\ref{sec:background_fdct}).
Therefore, it determines how finely the angles of the lines are resolved and
how sensitive to angular differences the descriptor is.

\begin{table}[h]
    \centering
    \pgfplotstableread[]{results/parameter_angles.csv}\resultsparameterangles
    \plottablexbars{scales,angles,cannysigma,metric}{\resultsparameterangles}
    \caption[Angle Parameter Results]{
        Influence of $N_{\theta}$ on the results of CANNY+PMEAN for $G=8$,
        $P=3$ and $\sigma=1.5$.
    }
    \label{tab:results_best_performers}
\end{table}

\FloatBarrier
%In an attempt to improve the
%result further, several sets of parameter values are examined: For the number
%of angles $N_{\theta} \in \{8, 12, 16\}$ the differences are small with
%$N_{\theta} = 12$ yielding the highest results. For the grid and patch sizes of
%the best results, $\frac{P}{G} \approx \frac{1}{3}$ seems to hold.


%The processing steps applied to the database images and the query images in the
%pipelines based on global features are almost identical, with the exception of
%the CANNY, SOBEL and SEGMENT steps for the database images. The curvelet
%responses are averaged on a grid and the features are ranked using the $L_2$
%and the COS distance measures.

%\paragraph{LUMA+MEAN}

%The most straightforward combination of processing steps consists of a LUMA
%input image, on which the means of $G \times G$ grid cells is computed for each
%scale and angle (Figure \ref{fig:pipeline_global_luma_mean}).  Varying $G$
%determines the feature size that can be encoded best. Setting $G=12$ seems to
%yield the best correlation, although the advantage over $G=8$ and $G=16$ is
%below $0.01$. The distance measure COS outperforms the $L_2$ measure (Table
%\ref{tab:results_global_luma_mean}). A possible explanation would be that the
%COS measure normalizes the magnitude of the feature vector as discussed in
%\ref{sec:anatomy_ranking_distance_euclidean}.

%\begin{figure}[h]
    %\centering
    %\input{illustrations/pipelines/global_luma_mean.tex}
    %\caption[Global LUMA+MEAN Pipelines]{
        %Global LUMA+MEAN Pipelines
    %}
    %\label{fig:pipeline_global_luma_mean}
%\end{figure}

%\begin{table}[h]
    %\centering
    %\pgfplotstableread[]{results/g_luma_mean_1.csv}\resultsglumamean
\begin{tikzpicture}
    \begin{axis}[
        hbarplot,
        yticklabels from table={\resultsglumamean}{Description},
        ]
        \addplot table[x=MeanCorrelation, y expr=\coordindex] {\resultsglumamean};
        \addreferenceplot;
    \end{axis}
\end{tikzpicture}

    %\caption[Global LUMA+MEAN Results]{
        %Global LUMA+MEAN Results
    %}
    %\label{tab:results_global_luma_mean}
%\end{table}

%\FloatBarrier
%\paragraph{CANNY+MEAN}

%In addition to reading the images like in the previous LUMA+MEAN configuration,
%this pipeline applies a CANNY processing step to the database images in an
%attempt to bring the query and database image domains closer together (Figure
%\ref{fig:pipeline_global_luma_canny_mean}). Again, the COS distance measure
%produces the best rankings (Table~\ref{tab:results_global_luma_canny_mean}).
%Surprisingly, the Canny edge detector does not lead to increased performance in
%comparison to plain the LUMA preprocessing step.

%\begin{figure}[h]
    %\centering
    %\begin{tikzpicture}[font=\tiny]
    \matrix[node grid] {
        \node [document node] (dbimg) {$I_{db}$}; &
        \node [operation node] (dbluma) {LUMA}; &
        \node [operation node] (dbcanny) {CANNY};  &
        \node [operation node] (dbcurvelet) {FDCT}; &
        \node [operation node] (dbmean) {MEAN}; \\
        \node [document node] (qimg) {$I_q$}; &
        \node [operation node] (qluma) {LUMA}; &&
        \node [operation node] (qcurvelet) {FDCT}; &
        \node [operation node] (qmean) {MEAN}; \\
    };

    \node [operation node, split node=2, right=3ex of $(dbmean.east)!0.5!(qmean.east)$] (dist) {$L_2$ \nodepart{two} COS};
    \node [document node, right=of dist] (result) {distances};

    \node [parameter node, above=of dbcanny] (dbcannyparam) {$\sigma$};
    \node [parameter node, above=of dbcurvelet] (dbcurveletparam) {$(N_s, N_{\theta})$};
    \node [parameter node, above=of dbmean] (dbmeanparam) {$G$};
    \node [parameter node, below=of qcurvelet] (qcurveletparam) {$(N_s, N_{\theta})$};
    \node [parameter node, below=of qmean] (qmeanparam) {$G$};

    \path [parameter connector] (dbcannyparam) -- (dbcanny);
    \path [parameter connector] (dbcurveletparam) -- (dbcurvelet);
    \path [parameter connector] (dbmeanparam) -- (dbmean);
    \path [parameter connector] (qcurveletparam) -- (qcurvelet);
    \path [parameter connector] (qmeanparam) -- (qmean);

    { [start chain=going right, every join/.style={connector}]
        \chainin (dbimg);
        \chainin (dbluma) [join];
        \chainin (dbcanny) [join];
        \chainin (dbcurvelet) [join];
        \chainin (dbmean) [join];
        \chainin (dist) [join=with dbmean.east by hvh connector top];
    }
    { [start chain=going right, every join/.style={connector}]
        \chainin (qimg);
        \chainin (qluma) [join];
        \chainin (qcurvelet) [join];
        \chainin (qmean) [join];
        \chainin (dist) [join=with qmean.east by hvh connector bottom];
        \chainin (result) [join];
    }
\end{tikzpicture}

    %\caption[Global CANNY+MEAN Pipelines]{
        %Global CANNY+MEAN Pipelines
    %}
    %\label{fig:pipeline_global_luma_canny_mean}
%\end{figure}

%\begin{table}[h]
    %\centering
    %\pgfplotstableread[]{results/g_luma_canny_mean_1.csv}\resultsglumacannymean
\plottablexbars{scales,angles,gridsize,cannysigma,metric}{\resultsglumacannymean}

    %\caption[Global CANNY+MEAN Results]{
        %Global CANNY+MEAN Results
    %}
    %\label{tab:results_global_luma_canny_mean}
%\end{table}

%\FloatBarrier
%\paragraph{SOBEL+MEAN}

%The SOBEL step used in this variant also attempts to bring the database images
%into the sketch domain (Figure~\ref{fig:pipeline_global_luma_sobel_mean}). The
%results are slightly better than with the CANNY preprocessor
%(Table~\ref{tab:results_global_luma_sobel_mean}).

%\begin{figure}[h!]
    %\centering
    %\input{illustrations/pipelines/global_luma_sobel_mean.tex}
    %\caption[Global SOBEL+MEAN Pipelines]{
        %Global SOBEL+MEAN Pipelines
    %}
    %\label{fig:pipeline_global_luma_sobel_mean}
%\end{figure}

%\begin{table}[h!]
    %\centering
    %\pgfplotstableread[]{results/g_luma_sobel_mean_1.csv}\resultsglumasobelmean
\begin{tikzpicture}
    \begin{axis}[
        hbarplot,
        yticklabels from table={\resultsglumasobelmean}{Description},
        ]
        \addplot table[x=MeanCorrelation, y expr=\coordindex] {\resultsglumasobelmean};
        \addreferenceplot;
    \end{axis}
\end{tikzpicture}

    %\caption[Global SOBEL+MEAN Results]{
        %Global SOBEL+MEAN Results
    %}
    %\label{tab:results_global_luma_sobel_mean}
%\end{table}

%\FloatBarrier
%\paragraph{SEGMENT+MEAN}

%With the gPb contour detector in the SEGMENT step to find edges in the database
%images (Figure~\ref{fig:pipeline_global_luma_segment_mean}), the $L_2$ distance
%metric produces results comparable to the COS metric in the CANNY+MEAN variant
%(Table~\ref{tab:results_global_luma_segment_mean}).  Unlike in the other cases,
%the COS distance measure performs worse than the $L_2$ metric.

%\begin{figure}[h]
    %\centering
    %\begin{tikzpicture}[font=\tiny]
    \matrix[node grid] {
        \node [document node] (dbimg) {$I_{db}$}; &
        \node [operation node] (dbsegment) {SEGMENT}; &
        \node [operation node] (dbcurvelet) {FDCT}; &
        \node [operation node] (dbmean) {MEAN}; \\
        \node [document node] (qimg) {$I_q$}; &
        \node [operation node] (qluma) {LUMA}; &
        \node [operation node] (qcurvelet) {FDCT}; &
        \node [operation node] (qmean) {MEAN}; \\
    };

    \node [operation node, split node=2, right=3ex of $(dbmean.east)!0.5!(qmean.east)$] (dist) {$L_2$ \nodepart{two} COS};
    \node [document node, right=of dist] (result) {distances};

    { [start chain=going right, every join/.style={connector}]
        \chainin (dbimg);
        \chainin (dbsegment) [join];
        \chainin (dbcurvelet) [join];
        \chainin (dbmean) [join];
        \chainin (dist) [join=with dbmean.east by hvh connector top];
    }
    { [start chain=going right, every join/.style={connector}]
        \chainin (qimg);
        \chainin (qluma) [join];
        \chainin (qcurvelet) [join];
        \chainin (qmean) [join];
        \chainin (dist) [join=with qmean.east by hvh connector bottom];
        \chainin (result) [join];
    }
\end{tikzpicture}

    %\caption[Global SEGMENT+MEAN Pipelines]{
        %Global SEGMENT+MEAN Pipelines
    %}
    %\label{fig:pipeline_global_luma_segment_mean}
%\end{figure}

%\begin{table}[h]
    %\centering
    %\pgfplotstableread[]{results/g_luma_segment_mean_1.csv}\resultsglumasegmentmean
\plottablexbars{scales,angles,gridsize,metric}{\resultsglumasegmentmean}

    %\caption[Global SEGMENT+MEAN Results]{
        %Global SEGMENT+MEAN Results
    %}
    %\label{tab:results_global_luma_segment_mean}
%\end{table}

%\FloatBarrier
