\subsection{Parameter Variations}\label{sec:results_parameters}

In an attempt to improve the results, the best performing configurations
presented in the previous sections will be re-used with varying parameter
values (\autoref{tab:results_best_performers}). The following sections compare
the results of changing the parameters $N_{\theta}$, $P$, $G$ and $\sigma$.

\begin{table}[h]
    \centering
    \pgfplotstableread[]{results/best_performers.csv}\resultsbestperformers
    \plottablexbars{imagereader,features,gridsize,patchsize,cannysigma,metric}{\resultsbestperformers}
    \caption[Best Performing Configurations]{
        Best Performing Configurations with default assumptions $N_s=4$ and
        $N_{\theta}=12$.
    }
    \label{tab:results_best_performers}
\end{table}

\FloatBarrier
\subsubsection{Curvelet Angles}

The parameter $N_{\theta}$ controls the number of angles, the curvelet
coronization is divided into (\autoref{sec:background_fdct}).  Therefore, it
determines how finely the angles of the lines are resolved and how sensitive to
angular differences the descriptor is.

As indicated by literature and exploratory experiments, an angular subdivision
of $N_{\theta} = 12$ appears to be optimal
(\autoref{tab:results_parameter_angles}). Larger values lead to worse, but
constant results.

\begin{table}[h]
    \centering
    \pgfplotstableread[]{results/parameter_angles.csv}\resultsparameterangles
    \plottablexbars{imagereader,features,scales,angles,metric}{\resultsparameterangles}
    \caption[Angle Parameter Results]{
        Influence of $N_{\theta}$ on the results of CANNY+PMEAN for $G=8$,
        $P=3$ and $\sigma=1.5$.
    }
    \label{tab:results_parameter_angles}
\end{table}

\FloatBarrier
\subsubsection{Grid and Patch Sizes}

All of the MEAN, PMEAN and PMEAN2 sampling methods use a regular grid to divide
the curvelet coefficients into cells, in which the mean of the coefficients is
calculated. Using a small number $G$ of subdivisions means that smaller
features might vanish within a large grid cell, unable to influence the mean
value. A finer subdivision allows for smaller features to be represented at the
risk of cutting apart larger features that lie on the grid lines. In addition
to $G$, the local sampling methods PMEAN and PMEAN2 are influenced by the
number of grid cells that make up a patch. As explained in
\autoref{sec:solution_signature_extraction_local}, a patch captures the
geometric relationships within a $P \times P$ neighborhood of cells. It thus
defines an upper limit on the size of a feature that can be represented
atomically. The results (\autoref{tab:results_parameter_grid}) indicate, that a
ratio of $\frac{P}{G} \approx \frac{1}{3}$ lead to a locally optimal solution.
This means, that features and their local composition in a neighborhood of
about $\frac{1}{3}$ of the image's width and height are best suited to
discriminate the images. This is similar to the $25\%$ optimum determined by
Eitz et al.\ \autocite{eitz_sketch-based_2011}.

\begin{table}[h]
    \centering
    \pgfplotstableread[]{results/parameter_grid.csv}\resultsparametergrid
    \plottablexbars{imagereader,features,gridsize,patchsize,metric}{\resultsparametergrid}
    \caption[Grid Size Parameter Results]{
        Influence of grid parameters $P$ and $G$ on the results for $N_s=4$,
        $N_{\theta}=12$ and $\sigma=1.5$.
    }
    \label{tab:results_parameter_grid}
\end{table}

\FloatBarrier
\subsubsection{Canny Sigma}

In the CANNY preprocessing step, the parameter $\sigma$ for the Gaussian
smoothing kernel can have a potentially large influence. It controls the spread
of the Gaussian distribution used for smoothing before the edge detection takes
place. Larger values lead to more smoothing, which makes the process less
dependent on image noise, but may cause the loss of important edge information.
The value $\sigma = 1.5$ yields the best correlation coefficients. Values
larger than $2$ tended to prevent any edge detection in images of the benchmark
dataset. Compared to the other parameters though, the influence appears to be
small except for the extreme values.

\begin{table}[h]
    \centering
    \pgfplotstableread[]{results/parameter_canny.csv}\resultsparametercanny
    \plottablexbars{imagereader,features,cannysigma,metric}{\resultsparametercanny}
    \caption[Canny Parameter Results]{
        Influence of the canny smoothing parameter $\sigma$ on the results for
        $N_s=4$, $N_{\theta}=12$, $G=8$ and $P=3$.
    }
    \label{tab:results_parameter_canny}
\end{table}

\FloatBarrier
