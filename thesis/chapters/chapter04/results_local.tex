\subsection{Local Features}

The local feature extraction pipelines utilize the same preprocessing steps as
the ones based on global features. The curvelet transform remains unchanged as
well. The the curvelet coefficients are sampled using the PMEAN and PMEAN2
strategy to produce features for local patches. From these feature vectors, a
codebook of size $1000$ is generated and the feature vectors from both the
database images and the query images are quantized and weighted using the
TF-ITF method. The metrics used for ranking are the generic cosine and $L_2$
distance measures as well as histogram intersection and the earth mover's
distance.

\paragraph{LUMA+PMEAN(2)}

Analogous to the global LUMA+MEAN variant, this pipeline configuration directly
uses the LUMA image representations as input for the curvelet transform
(Figure~\ref{fig:pipeline_local_luma_pmean}). For both, the PMEAN
(Table~\ref{tab:results_local_luma_pmean}) and PMEAN2
(Table~\ref{tab:results_local_luma_pmean2}) sampling method, histogram
intersection and the cosine measures are far superior to the $L_2$ metric and
even the EMD metric, although the results cannot compete with the global
LUMA+MEAN version.

\begin{figure}[h]
    \centering
    \begin{tikzpicture}[font=\tiny]
    \matrix[node grid] {
        \node [document node] (dbimg) {$I_{db}$}; &
        \node [operation node] (dbluma) {LUMA}; &
        \node [operation node] (dbcurvelet) {FDCT}; &
        %\node [operation node] (dbpmean) {PMEAN}; &&
        \node [operation node, split node=2] (dbpmean) {PMEAN \nodepart{two} PMEAN2}; &
        \node [operation node] (dbquant) {VQ}; \\
        &&&
        \node [operation node] (cluster) {CLUSTER}; \\
        \node [document node] (qimg) {$I_q$}; &
        \node [operation node] (qluma) {LUMA}; &
        \node [operation node] (qcurvelet) {FDCT}; &
        %\node [operation node] (qpmean) {PMEAN}; &&
        \node [operation node, split node=2] (qpmean) {PMEAN \nodepart{two} PMEAN2}; &
        \node [operation node] (qquant) {VQ}; \\
    };
    \node [operation node, split node=4, right=3ex of $(dbquant.east)!0.5!(qquant.east)$] (dist) {$L_2$ \nodepart{two} COS \nodepart{three} HI(B) \nodepart{four} EMD};
    \node [document node, right=of dist] (result) {distances};

    { [start chain=going right, every join/.style={connector}]
        \chainin (dbimg);
        \chainin (dbluma) [join];
        \chainin (dbcurvelet) [join];
        \chainin (dbpmean) [join];
        { [start branch]
            \chainin (cluster) [join];
            { [start branch]
                \chainin (qquant) [join=with cluster.-5 by hv connector];
            }
            \chainin (dbquant) [join=with cluster.5 by hv connector];
        }
        \chainin (dbquant) [join];
        \chainin (dist) [join=with dbquant.east by hvh connector top];
    }
    { [start chain=going right, every join/.style={connector}]
        \chainin (qimg);
        \chainin (qluma) [join];
        \chainin (qcurvelet) [join];
        \chainin (qpmean) [join];
        \chainin (qquant) [join];
        \chainin (dist) [join=with qquant.east by hvh connector bottom];
        \chainin (result) [join];
    }
\end{tikzpicture}

    \caption[Local LUMA+PMEAN(2) Pipelines]{
        Local LUMA+PMEAN(2) Pipelines
    }
    \label{fig:pipeline_local_luma_pmean}
\end{figure}

\begin{table}[h]
    \centering
    \subfloat[Local LUMA+PMEAN Results]{%
        \pgfplotstableread[]{results/l_luma_pmean_1.csv}\resultsllumapmean
\plottablexbars{scales,angles,gridsize,patchsize,metric}{\resultsllumapmean}
%\begin{tikzpicture}
    %\begin{axis}[
        %hbarplot,
        %yticklabels from table={\resultsllumacannypmean}{Description},
        %%width=0.8\textwidth,
        %%height=0.2\textheight,
        %]
        %\addplot table[x=MeanCorrelation, y expr=\coordindex] {\resultsllumacannypmean};
        %\addreferenceplot;
    %\end{axis}
%\end{tikzpicture}

        \label{tab:results_local_luma_pmean}
    }
    \quad
    \subfloat[Local LUMA+PMEAN2 Results]{%
        \pgfplotstableread[]{results/l_luma_pmean2_1.csv}\resultsllumapmeantwo
\plottablexbars{scales,angles,gridsize,patchsize,metric}{\resultsllumapmeantwo}
%\begin{tikzpicture}
    %\begin{axis}[
        %hbarplot,
        %yticklabels from table={\resultsllumacannypmean}{Description},
        %%width=0.8\textwidth,
        %%height=0.2\textheight,
        %]
        %\addplot table[x=MeanCorrelation, y expr=\coordindex] {\resultsllumacannypmean};
        %\addreferenceplot;
    %\end{axis}
%\end{tikzpicture}

        \label{tab:results_local_luma_pmean2}
    }
    \caption[Local LUMA+PMEAN(2) Results]{
        Local LUMA+PMEAN(2) Results
    }
    \label{tab:results_local_luma_pmean_all}
\end{table}

\paragraph{CANNY+PMEAN(2)}

Using the Canny edge detector and PMEAN sampling
(Figure~\ref{fig:pipeline_local_luma_canny_pmean}), the rank correlation
coefficient for COS and HI exceed all previous results
(Table~\ref{tab:results_local_luma_canny_pmean}). In an attempt to improve the
result further, several sets of parameter values are examined: For the number
of angles $N_{\theta} \in \{8, 12, 16\}$ the differences are small with
$N_{\theta} = 12$ yielding the highest results. For the grid and patch sizes of
the best results, $\frac{P}{G} \approx \frac{1}{3}$ seems to hold.

Treating the coefficients on different scales separately using the PMEAN2
sampling method is inferior to PMEAN in this case.

\begin{figure}[h!]
    \centering
    \begin{tikzpicture}[font=\tiny]
    \matrix[node grid] {
        \node [document node] (dbimg) {$I_{db}$}; &
        \node [operation node] (dbluma) {LUMA}; &
        \node [operation node] (dbcanny) {CANNY}; &
        \node [operation node] (dbcurvelet) {FDCT}; &
        %\node [operation node] (dbpmean) {PMEAN}; &&
        \node [operation node, split node=2] (dbpmean) {PMEAN \nodepart{two} PMEAN2}; &
        \node [operation node] (dbquant) {VQ}; \\
        &&&&
        \node [operation node] (cluster) {CLUSTER}; \\
        \node [document node] (qimg) {$I_q$}; &
        \node [operation node] (qluma) {LUMA}; &&
        \node [operation node] (qcurvelet) {FDCT}; &
        %\node [operation node] (qpmean) {PMEAN}; &&
        \node [operation node, split node=2] (qpmean) {PMEAN \nodepart{two} PMEAN2}; &
        \node [operation node] (qquant) {VQ}; \\
    };
    \node [operation node, split node=4, right=3ex of $(dbquant.east)!0.5!(qquant.east)$] (dist) {$L_2$ \nodepart{two} COS \nodepart{three} HI(B) \nodepart{four} EMD};
    \node [document node, right=of dist] (result) {distances};

    \node [parameter node, above=of dbcanny] (dbcannyparam) {$\sigma$};
    \node [parameter node, above=of dbcurvelet] (dbcurveletparam) {$(N_s, N_{\theta})$};
    \node [parameter node, above=of dbpmean] (dbpmeanparam) {$(G, P)$};
    \node [parameter node, below=of qcurvelet] (qcurveletparam) {$(N_s, N_{\theta})$};
    \node [parameter node, below=of qpmean] (qpmeanparam) {$(G, P)$};

    \path [parameter connector] (dbcannyparam) -- (dbcanny);
    \path [parameter connector] (dbcurveletparam) -- (dbcurvelet);
    \path [parameter connector] (dbpmeanparam) -- (dbpmean);
    \path [parameter connector] (qcurveletparam) -- (qcurvelet);
    \path [parameter connector] (qpmeanparam) -- (qpmean);

    { [start chain=going right, every join/.style={connector}]
        \chainin (dbimg);
        \chainin (dbluma) [join];
        \chainin (dbcanny) [join];
        \chainin (dbcurvelet) [join];
        \chainin (dbpmean) [join];
        { [start branch]
            \chainin (cluster) [join];
            { [start branch]
                \chainin (qquant) [join=with cluster.-5 by hv connector];
            }
            \chainin (dbquant) [join=with cluster.5 by hv connector];
        }
        \chainin (dbquant) [join];
        \chainin (dist) [join=with dbquant.east by hvh connector top];
    }
    { [start chain=going right, every join/.style={connector}]
        \chainin (qimg);
        \chainin (qluma) [join];
        \chainin (qcurvelet) [join];
        \chainin (qpmean) [join];
        \chainin (qquant) [join];
        \chainin (dist) [join=with qquant.east by hvh connector bottom];
        \chainin (result) [join];
    }
\end{tikzpicture}

    \caption[Local CANNY+PMEAN Pipelines]{
        Local CANNY+PMEAN Pipelines
    }
    \label{fig:pipeline_local_luma_canny_pmean}
\end{figure}

\begin{table}[h!]
    \centering
    \subfloat[Local CANNY+PMEAN Results]{%
        \pgfplotstableread[]{results/l_luma_canny_pmean_1.csv}\resultsllumacannypmean
\begin{tikzpicture}
    \begin{axis}[
        hbarplot,
        yticklabels from table={\resultsllumacannypmean}{Description},
        %width=0.8\textwidth,
        %height=0.2\textheight,
        ]
        \addplot table[x=MeanCorrelation, y expr=\coordindex] {\resultsllumacannypmean};
        \addreferenceplot;
    \end{axis}
\end{tikzpicture}

        \label{tab:results_local_luma_canny_pmean}
    }
    \quad
    \subfloat[Local CANNY+PMEAN2 Results]{%
        \pgfplotstableread[]{results/l_luma_canny_pmean2_1.csv}\resultsllumacannypmeantwo
\plottablexbars{scales,angles,gridsize,patchsize,cannysigma,metric}{\resultsllumacannypmeantwo}
%\begin{tikzpicture}
    %\begin{axis}[
        %hbarplot,
        %yticklabels from table={\resultsllumacannypmean}{Description},
        %%width=0.8\textwidth,
        %%height=0.2\textheight,
        %]
        %\addplot table[x=MeanCorrelation, y expr=\coordindex] {\resultsllumacannypmean};
        %\addreferenceplot;
    %\end{axis}
%\end{tikzpicture}

        \label{tab:results_local_luma_canny_pmean2}
    }
    \caption[Local CANNY+PMEAN(2) Results]{
        Local CANNY+PMEAN(2) Results
    }
    \label{tab:results_local_luma_canny_pmean_all}
\end{table}

\paragraph{SOBEL+PMEAN(2)}

foo

\begin{figure}[h]
    \centering
    \begin{tikzpicture}[font=\tiny]
    \matrix[node grid] {
        \node [document node] (dbimg) {$I_{db}$}; &
        \node [operation node] (dbluma) {LUMA}; &
        \node [operation node] (dbsobel) {SOBEL}; &
        \node [operation node] (dbcurvelet) {FDCT}; &
        %\node [operation node] (dbpmean) {PMEAN}; &&
        \node [operation node, split node=2] (dbpmean) {PMEAN \nodepart{two} PMEAN2}; &
        \node [operation node] (dbquant) {VQ}; \\
        &&&&
        \node [operation node] (cluster) {CLUSTER}; \\
        \node [document node] (qimg) {$I_q$}; &
        \node [operation node] (qluma) {LUMA}; &&
        \node [operation node] (qcurvelet) {FDCT}; &
        %\node [operation node] (qpmean) {PMEAN}; &&
        \node [operation node, split node=2] (qpmean) {PMEAN \nodepart{two} PMEAN2}; &
        \node [operation node] (qquant) {VQ}; \\
    };
    \node [operation node, split node=4, right=3ex of $(dbquant.east)!0.5!(qquant.east)$] (dist) {$L_2$ \nodepart{two} COS \nodepart{three} HI(B) \nodepart{four} EMD};
    \node [document node, right=of dist] (result) {distances};

    { [start chain=going right, every join/.style={connector}]
        \chainin (dbimg);
        \chainin (dbluma) [join];
        \chainin (dbsobel) [join];
        \chainin (dbcurvelet) [join];
        \chainin (dbpmean) [join];
        { [start branch]
            \chainin (cluster) [join];
            { [start branch]
                \chainin (qquant) [join=with cluster.-5 by hv connector];
            }
            \chainin (dbquant) [join=with cluster.5 by hv connector];
        }
        \chainin (dbquant) [join];
        \chainin (dist) [join=with dbquant.east by hvh connector top];
    }
    { [start chain=going right, every join/.style={connector}]
        \chainin (qimg);
        \chainin (qluma) [join];
        \chainin (qcurvelet) [join];
        \chainin (qpmean) [join];
        \chainin (qquant) [join];
        \chainin (dist) [join=with qquant.east by hvh connector bottom];
        \chainin (result) [join];
    }
\end{tikzpicture}

    \caption[Local SOBEL+PMEAN Pipelines]{
        Local SOBEL+PMEAN Pipelines
    }
    \label{fig:pipeline_local_luma_sobel_pmean}
\end{figure}

\begin{table}[h]
    \centering
    \subfloat[Local SOBEL+PMEAN Results]{%
        \pgfplotstableread[]{results/l_luma_sobel_pmean_1.csv}\resultsllumasobelpmean
\plottablexbars{scales,angles,gridsize,patchsize,metric}{\resultsllumasobelpmean}

        \label{tab:results_local_luma_sobel_pmean}
    }
    \quad
    \subfloat[Local SOBEL+PMEAN2 Results]{%
        \pgfplotstableread[]{results/l_luma_sobel_pmean2_1.csv}\resultsllumasobelpmeantwo
\plottablexbars{scales,angles,gridsize,patchsize,metric}{\resultsllumasobelpmeantwo}

        \label{tab:results_local_luma_sobel_pmean2}
    }
    \caption[Local SOBEL+PMEAN(2) Results]{
        Local SOBEL+PMEAN(2) Results
    }
    \label{tab:results_local_luma_sobel_pmean_all}
\end{table}

\paragraph{SEGMENT+PMEAN(2)}

foo

\begin{figure}[h]
    \centering
    \begin{tikzpicture}[font=\tiny]
    \matrix[node grid] {
        \node [document node] (dbimg) {$I_{db}$}; &
        \node [operation node] (dbluma) {LUMA}; &
        \node [operation node] (dbsegment) {SEGMENT}; &
        \node [operation node] (dbcurvelet) {FDCT}; &
        %\node [operation node] (dbpmean) {PMEAN}; &&
        \node [operation node, split node=2] (dbpmean) {PMEAN \nodepart{two} PMEAN2}; &
        \node [operation node] (dbquant) {VQ}; \\
        &&&&
        \node [operation node] (cluster) {CLUSTER}; \\
        \node [document node] (qimg) {$I_q$}; &
        \node [operation node] (qluma) {LUMA}; &&
        \node [operation node] (qcurvelet) {FDCT}; &
        %\node [operation node] (qpmean) {PMEAN}; &&
        \node [operation node, split node=2] (qpmean) {PMEAN \nodepart{two} PMEAN2}; &
        \node [operation node] (qquant) {VQ}; \\
    };
    \node [operation node, split node=4, right=3ex of $(dbquant.east)!0.5!(qquant.east)$] (dist) {$L_2$ \nodepart{two} COS \nodepart{three} HI(B) \nodepart{four} EMD};
    \node [document node, right=of dist] (result) {distances};

    { [start chain=going right, every join/.style={connector}]
        \chainin (dbimg);
        \chainin (dbluma) [join];
        \chainin (dbsegment) [join];
        \chainin (dbcurvelet) [join];
        \chainin (dbpmean) [join];
        { [start branch]
            \chainin (cluster) [join];
            { [start branch]
                \chainin (qquant) [join=with cluster.-5 by hv connector];
            }
            \chainin (dbquant) [join=with cluster.5 by hv connector];
        }
        \chainin (dbquant) [join];
        \chainin (dist) [join=with dbquant.east by hvh connector top];
    }
    { [start chain=going right, every join/.style={connector}]
        \chainin (qimg);
        \chainin (qluma) [join];
        \chainin (qcurvelet) [join];
        \chainin (qpmean) [join];
        \chainin (qquant) [join];
        \chainin (dist) [join=with qquant.east by hvh connector bottom];
        \chainin (result) [join];
    }
\end{tikzpicture}

    \caption[Local SEGMENT+PMEAN Pipelines]{
        Local SEGMENT+PMEAN Pipelines
    }
    \label{fig:pipeline_local_luma_segment_pmean}
\end{figure}

\begin{table}[h]
    \centering
    \subfloat[Local SEGMENT+PMEAN Results]{
        \pgfplotstableread[]{results/l_luma_segment_pmean_1.csv}\resultsllumasegmentpmean
\plottablexbars{scales,angles,gridsize,patchsize,metric}{\resultsllumasegmentpmean}

        \label{tab:results_local_luma_segment_pmean}
    }
    \quad
    \subfloat[Local SEGMENT+PMEAN2 Results]{
        \pgfplotstableread[]{results/l_luma_segment_pmean2_1.csv}\resultsllumasegmentpmeantwo
\plottablexbars{scales,angles,gridsize,patchsize,metric}{\resultsllumasegmentpmeantwo}

        \label{tab:results_local_luma_segment_pmean2}
    }
    \caption[Local SEGMENT+PMEAN(2) Results]{
        Local SEGMENT+PMEAN(2) Results
    }
    \label{tab:results_local_luma_segment_pmean_all}
\end{table}
