\subsection{Global Features}

\paragraph{LUMA+MEAN}

The most straightforward combination of processing steps consists of a LUMA
input image, on which the means of $G \times G$ grid cells is computed for each
scale and angle (Figure \ref{fig:pipeline_global_luma_mean_l2} and
\ref{fig:pipeline_global_luma_mean_cos}). Varying $G$ determines the feature
size that can be encoded best. Setting $G=12$ seems to yield the best
correlation, although the advantage over $G=8$ and $G=16$ is below $0.01$. The
distance measure COS outperforms the $L_2$ measure (Table
\ref{tab:results_global_luma_mean}). A possible explanation would be that the
COS measure normalizes the magnitude of the feature vector as discussed in
\ref{sec:anatomy_ranking_distance_euclidean}.

\begin{figure}[h]
    \centering
    \subfloat[LUMA+MEAN+$L_2$ Pipeline]{%
        \input{illustrations/pipelines/global_luma_mean_l2.tex}
        \label{fig:pipeline_global_luma_mean_l2}
    }
    \quad
    \subfloat[LUMA+MEAN+COS Pipeline]{%
        \input{illustrations/pipelines/global_luma_mean_cos.tex}
        \label{fig:pipeline_global_luma_mean_cos}
    }
    %\quad
    %\subfloat[LUMA+MEAN Results]{%
        %\pgfplotstableread[]{results/g_luma_mean_1.csv}\resultsglumamean
\begin{tikzpicture}
    \begin{axis}[
        hbarplot,
        yticklabels from table={\resultsglumamean}{Description},
        ]
        \addplot table[x=MeanCorrelation, y expr=\coordindex] {\resultsglumamean};
        \addreferenceplot;
    \end{axis}
\end{tikzpicture}

        %\label{fig:results_global_luma_mean}
    %}
    \caption[Global LUMA+MEAN Pipelines]{
        Global LUMA+MEAN Pipelines
    }
    \label{fig:pipeline_global_luma_mean}
\end{figure}

\begin{table}[h]
    \pgfplotstableread[]{results/g_luma_mean_1.csv}\resultsglumamean
\begin{tikzpicture}
    \begin{axis}[
        hbarplot,
        yticklabels from table={\resultsglumamean}{Description},
        ]
        \addplot table[x=MeanCorrelation, y expr=\coordindex] {\resultsglumamean};
        \addreferenceplot;
    \end{axis}
\end{tikzpicture}

    \caption[Global LUMA+MEAN Results]{
        Global LUMA+MEAN Results
    }
    \label{tab:results_global_luma_mean}
\end{table}

\paragraph{CANNY+MEAN}

In addition to reading the images like in the previous LUMA+MEAN configuration,
this pipeline applies a CANNY processing step to the database images in an
attempt to bring the query and database image domains closer together (Figure
\ref{fig:pipeline_global_luma_canny_mean_l2} and
\ref{fig:pipeline_global_luma_canny_mean_cos}). Again, the COS distance measure
produces the best rankings (Table~\ref{tab:results_global_luma_canny_mean}).
Surprisingly, the Canny edge detector does not lead to increased performance in
comparison to plain the LUMA preprocessing step.

\begin{figure}[h]
    \centering
    \subfloat[CANNY+MEAN+$L_2$ Pipeline]{%
        \begin{tikzpicture}[font=\tiny]
    \matrix[node grid] {
        \node [document node] (dbimg) {$I_{db}$}; &
        \node [operation node] (dbluma) {LUMA}; &
        \node [operation node] (dbcanny) {CANNY};  &
        \node [operation node] (dbcurvelet) {FDCT}; &
        \node [operation node] (dbmean) {MEAN}; \\
        \node [document node] (qimg) {$I_q$}; &
        \node [operation node] (qluma) {LUMA}; &&
        \node [operation node] (qcurvelet) {FDCT}; &
        \node [operation node] (qmean) {MEAN}; \\
    };

    \node [operation node, split node=2, right=3ex of $(dbmean.east)!0.5!(qmean.east)$] (dist) {$L_2$ \nodepart{two} COS};
    \node [document node, right=of dist] (result) {distances};

    \node [parameter node, above=of dbcanny] (dbcannyparam) {$\sigma$};
    \node [parameter node, above=of dbcurvelet] (dbcurveletparam) {$(N_s, N_{\theta})$};
    \node [parameter node, above=of dbmean] (dbmeanparam) {$G$};
    \node [parameter node, below=of qcurvelet] (qcurveletparam) {$(N_s, N_{\theta})$};
    \node [parameter node, below=of qmean] (qmeanparam) {$G$};

    \path [parameter connector] (dbcannyparam) -- (dbcanny);
    \path [parameter connector] (dbcurveletparam) -- (dbcurvelet);
    \path [parameter connector] (dbmeanparam) -- (dbmean);
    \path [parameter connector] (qcurveletparam) -- (qcurvelet);
    \path [parameter connector] (qmeanparam) -- (qmean);

    { [start chain=going right, every join/.style={connector}]
        \chainin (dbimg);
        \chainin (dbluma) [join];
        \chainin (dbcanny) [join];
        \chainin (dbcurvelet) [join];
        \chainin (dbmean) [join];
        \chainin (dist) [join=with dbmean.east by hvh connector top];
    }
    { [start chain=going right, every join/.style={connector}]
        \chainin (qimg);
        \chainin (qluma) [join];
        \chainin (qcurvelet) [join];
        \chainin (qmean) [join];
        \chainin (dist) [join=with qmean.east by hvh connector bottom];
        \chainin (result) [join];
    }
\end{tikzpicture}

        \label{fig:pipeline_global_luma_canny_mean_l2}
    }
    \quad
    \subfloat[CANNY+MEAN+COS Pipeline]{%
        \begin{tikzpicture}[font=\tiny]
    \matrix[node grid] {
        \node [document node] (dbimg) {$I_{db}$}; &
        \node [operation node] (dbluma) {LUMA}; &
        \node [operation node] (dbcanny) {CANNY};  &
        \node [operation node] (dbcurvelet) {FDCT}; &
        \node [operation node] (dbmean) {MEAN}; \\
        \node [document node] (qimg) {$I_q$}; &
        \node [operation node] (qluma) {LUMA}; &&
        \node [operation node] (qcurvelet) {FDCT}; &
        \node [operation node] (qmean) {MEAN}; &
        \node [operation node] (dist) {COS}; &
        \node [document node] (result) {distances}; \\
    };
    { [start chain=going right, every join/.style={connector}]
        \chainin (dbimg);
        \chainin (dbluma) [join];
        \chainin (dbcanny) [join];
        \chainin (dbcurvelet) [join];
        \chainin (dbmean) [join];
        \chainin (dist) [join=by hv connector];
    }
    { [start chain=going right, every join/.style={connector}]
        \chainin (qimg);
        \chainin (qluma) [join];
        \chainin (qcurvelet) [join];
        \chainin (qmean) [join];
        \chainin (dist) [join];
        \chainin (result) [join];
    }
\end{tikzpicture}

        \label{fig:pipeline_global_luma_canny_mean_cos}
    }
    %\quad
    %\subfloat[CANNY+MEAN Results]{%
        %\pgfplotstableread[]{results/g_luma_canny_mean_1.csv}\resultsglumacannymean
\plottablexbars{scales,angles,gridsize,cannysigma,metric}{\resultsglumacannymean}

        %\label{fig:results_global_luma_canny_mean}
    %}
    \caption[Global CANNY+MEAN Pipelines]{
        Global CANNY+MEAN Pipelines
    }
    \label{fig:pipeline_global_luma_canny_mean}
\end{figure}

\begin{table}[h]
    \pgfplotstableread[]{results/g_luma_canny_mean_1.csv}\resultsglumacannymean
\plottablexbars{scales,angles,gridsize,cannysigma,metric}{\resultsglumacannymean}

    \caption[Global CANNY+MEAN Results]{
        Global CANNY+MEAN Results
    }
    \label{tab:results_global_luma_canny_mean}
\end{table}

\paragraph{SOBEL+MEAN}

The SOBEL step used in this variant also attempts to bring the database images
into the sketch domain. The results are even slightly better than with the
CANNY preprocessor.

\begin{figure}[h]
    \centering
    \subfloat[SOBEL+MEAN+$L_2$ Pipeline]{%
        \input{illustrations/pipelines/global_luma_sobel_mean_l2.tex}
        \label{fig:pipeline_global_luma_sobel_mean_l2}
    }
    \quad
    \subfloat[SOBEL+MEAN+COS Pipeline]{%
        \input{illustrations/pipelines/global_luma_sobel_mean_cos.tex}
        \label{fig:pipeline_global_luma_sobel_mean_cos}
    }
    %\quad
    %\subfloat[SOBEL+MEAN Results]{%
        %\pgfplotstableread[]{results/g_luma_sobel_mean_1.csv}\resultsglumasobelmean
\begin{tikzpicture}
    \begin{axis}[
        hbarplot,
        yticklabels from table={\resultsglumasobelmean}{Description},
        ]
        \addplot table[x=MeanCorrelation, y expr=\coordindex] {\resultsglumasobelmean};
        \addreferenceplot;
    \end{axis}
\end{tikzpicture}

        %\label{fig:results_global_luma_sobel_mean}
    %}
    \caption[Global SOBEL+MEAN Pipelines]{
        Global SOBEL+MEAN Pipelines
    }
    \label{fig:pipeline_global_luma_sobel_mean}
\end{figure}

\begin{table}[h]
    \pgfplotstableread[]{results/g_luma_sobel_mean_1.csv}\resultsglumasobelmean
\begin{tikzpicture}
    \begin{axis}[
        hbarplot,
        yticklabels from table={\resultsglumasobelmean}{Description},
        ]
        \addplot table[x=MeanCorrelation, y expr=\coordindex] {\resultsglumasobelmean};
        \addreferenceplot;
    \end{axis}
\end{tikzpicture}

    \caption[Global SOBEL+MEAN Results]{
        Global SOBEL+MEAN Results
    }
    \label{tab:results_global_luma_sobel_mean}
\end{table}
