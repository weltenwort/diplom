\section{Benchmarking Method}\label{sec:results_benchmarking}

A usual way to evaluate the performance of retrieval systems is to calculate
the ratio of true positive and false positive matches and visualize it in a
receiver operating characteristic (ROC) curve. While that approach is well
suited for benchmarking binary decision algorithms, it is not appropriate for
retrieval problems, that do not feature a well-defined "correct" solution. An
alternative approach is looking at the recall and precision characteristics
defined as
\begin{align*}
    recall & = \frac{\text{number of correct positive results}}{\text{total number of positives}} \\
    precision & = \frac{\text{number of correct positive results}}{\text{total number of results}}
\end{align*}
Even though this metric works better for algorithms that return a set of
results, it is still based on the notion of a "positive match", which requires
an a priori classification of the benchmark data. For intra-domain evaluations
the sketch dataset created by Eitz et al.\ \autocite{eitz_how_2012} is used.
It consists of 20.000 hand-drawn sketches obtained via crowd-sourcing, that are
evenly divided into 250 categories. To speed up computations, 50 of those
categories are chosen to derive precision-recall statistics. From each
category, an image is randomly selected as the query and the rest is used as
positive results. In this case, both the query images and the database images
are from the sketch domain, so the effectiveness of the retrieval process
without preprocessing biases can be examined.

Since sketch-based image retrieval systems are most likely to be used in
interactive search applications of some form, it is desirable to assess the
performance in relation to the results a human would achieve.  Therefore the
benchmark used to evaluate the retrieval pipelines in cross-domain applications
corresponds to the method described by Eitz et al.\
\autocite{eitz_sketch-based_2011}, in which the authors create a benchmark
dataset and perform a user study with 28 participants to define "ground truth"
rankings. The dataset is divided into 31 groups of one sketch and 40 images
each. Participants ranked the 40 images within each group by assigning scores
indicating the similarity to the corresponding sketch in a controlled study
environment. Each sketch/image pair's final ground truth ranking is calculated
as the mean of the scores assigned by all participants.

To compare a ground truth ranking $x = (x_1, x_2, \dots, x_n)$ to a ranking
$y = (y_1, y_2, \dots, y_n)$ produced by a retrieval system, the Kendall
rank correlation coefficient $\tau_B$ is used. It measures the similarity of
the orderings by grouping all pairs $p_{i, j} = \{(x_i, y_i), (x_j, y_j)\}$,
$i, j \in 1, \dots, n$ into 5 sets:
\begin{align*}
    p_{i, j} & \in C & \quad\text{if } x_i < x_j \text{ and } y_i < y_j \\
    p_{i, j} & \in D & \quad\text{if } x_i < x_j \text{ and } y_i > y_j \\
    p_{i, j} & \in T_x & \quad\text{if } x_i = x_j \text{ and } y_i \neq y_j \\
    p_{i, j} & \in T_y & \quad\text{if } x_i \neq x_j \text{ and } y_i = y_j \\
    p_{i, j} & \in T_{xy} & \quad\text{if } x_i = x_j \text{ and } y_i = y_j
\end{align*}
From that, the correlation value $\tau_B$ in the interval $[-1, 1]$ can be
calculated as
\begin{equation*}
    \tau_B = \frac{|C| - |D|}{\sqrt{(|C| + |D| + |T_x|)(|C| + |D| + |T_y|)}}.
\end{equation*}
The higher $\tau_B$ is, the more pairs in $x$ and $y$ have a similar ordering.
Since the values are only compared within each ranking, the result is
independent of each rankings' scaling, making it ideal for comparison of
different distance metrics.
