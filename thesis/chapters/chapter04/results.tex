\section{Variants and Results}\label{sec:results}

\subsection{Global Features}

\paragraph{LUMA+MEAN}

The most straightforward combination of processing steps consists of a LUMA
input image, on which the means of $G \times G$ grid cells is computed for each
scale and angle (Figure \ref{fig:pipeline_global_luma_mean_l2} and
\ref{fig:pipeline_global_luma_mean_cos}). Varying $G$ appears to have a
negligible influence on the results. The distance measure COS outperforms the
$L_2$ measure (Figure \ref{fig:results_global_luma_mean}). A possible
explanation would be that the COS measure normalizes the magnitude of the
feature vector as discussed in \ref{sec:anatomy_ranking_distance_euclidean}.

\begin{figure}[h]
    \centering
    \subfloat[LUMA+MEAN+$L_2$ Pipeline]{%
        \input{illustrations/pipelines/global_luma_mean_l2.tex}
        \label{fig:pipeline_global_luma_mean_l2}
    }
    \quad
    \subfloat[LUMA+MEAN+COS Pipeline]{%
        \input{illustrations/pipelines/global_luma_mean_cos.tex}
        \label{fig:pipeline_global_luma_mean_cos}
    }
    \quad
    \subfloat[LUMA+MEAN Results]{%
        \pgfplotstableread[]{results/g_luma_mean_1.csv}\resultsglumamean
\begin{tikzpicture}
    \begin{axis}[
        hbarplot,
        yticklabels from table={\resultsglumamean}{Description},
        ]
        \addplot table[x=MeanCorrelation, y expr=\coordindex] {\resultsglumamean};
        \addreferenceplot;
    \end{axis}
\end{tikzpicture}

        \label{fig:results_global_luma_mean}
    }
    \caption[Global LUMA+MEAN Pipelines and Results]{
        Global LUMA+MEAN Pipelines and Results
    }
    \label{fig:pipeline_global_luma_mean}
\end{figure}

\paragraph{CANNY+MEAN}

foo

\begin{figure}[h]
    \centering
    \subfloat[CANNY+MEAN+$L_2$ Pipeline]{%
        \input{illustrations/pipelines/global_luma_canny_mean_l2.tex}
        \label{fig:pipeline_global_luma_canny_mean_l2}
    }
    \quad
    \subfloat[CANNY+MEAN+COS Pipeline]{%
        \begin{tikzpicture}[font=\tiny]
    \matrix[node grid] {
        \node [document node] (dbimg) {$I_{db}$}; &
        \node [operation node] (dbluma) {LUMA}; &
        \node [operation node] (dbcanny) {CANNY};  &
        \node [operation node] (dbcurvelet) {FDCT}; &
        \node [operation node] (dbmean) {MEAN}; \\
        \node [document node] (qimg) {$I_q$}; &
        \node [operation node] (qluma) {LUMA}; &&
        \node [operation node] (qcurvelet) {FDCT}; &
        \node [operation node] (qmean) {MEAN}; &
        \node [operation node] (dist) {COS}; &
        \node [document node] (result) {distances}; \\
    };
    { [start chain=going right, every join/.style={connector}]
        \chainin (dbimg);
        \chainin (dbluma) [join];
        \chainin (dbcanny) [join];
        \chainin (dbcurvelet) [join];
        \chainin (dbmean) [join];
        \chainin (dist) [join=by hv connector];
    }
    { [start chain=going right, every join/.style={connector}]
        \chainin (qimg);
        \chainin (qluma) [join];
        \chainin (qcurvelet) [join];
        \chainin (qmean) [join];
        \chainin (dist) [join];
        \chainin (result) [join];
    }
\end{tikzpicture}

        \label{fig:pipeline_global_luma_canny_mean_cos}
    }
    \quad
    \subfloat[CANNY+MEAN Results]{%
        \pgfplotstableread[]{results/g_luma_canny_mean_1.csv}\resultsglumacannymean
\plottablexbars{scales,angles,gridsize,cannysigma,metric}{\resultsglumacannymean}

        \label{fig:results_global_luma_canny_mean}
    }
    \caption[Global CANNY+MEAN Pipelines and Results]{
        Global CANNY+MEAN Pipelines and Results
    }
    \label{fig:pipeline_global_luma_canny_mean}
\end{figure}

\paragraph{SOBEL+MEAN}

In addition to reading the images like in the previous LUMA+MEAN configuration,
this pipeline applies a SOBEL processing step to the database images in an
attempt to bring the query and database image domains closer together (Figure
\ref{fig:pipeline_global_luma_mean_l2} and
\ref{fig:pipeline_global_luma_mean_cos}).

\begin{figure}[h]
    \centering
    \subfloat[SOBEL+MEAN+$L_2$ Pipeline]{%
        \input{illustrations/pipelines/global_luma_sobel_mean_l2.tex}
        \label{fig:pipeline_global_luma_sobel_mean_l2}
    }
    \quad
    \subfloat[SOBEL+MEAN+COS Pipeline]{%
        \input{illustrations/pipelines/global_luma_sobel_mean_cos.tex}
        \label{fig:pipeline_global_luma_sobel_mean_cos}
    }
    \quad
    \subfloat[SOBEL+MEAN Results]{%
        \pgfplotstableread[]{results/g_luma_sobel_mean_1.csv}\resultsglumasobelmean
\begin{tikzpicture}
    \begin{axis}[
        hbarplot,
        yticklabels from table={\resultsglumasobelmean}{Description},
        ]
        \addplot table[x=MeanCorrelation, y expr=\coordindex] {\resultsglumasobelmean};
        \addreferenceplot;
    \end{axis}
\end{tikzpicture}

        \label{fig:results_global_luma_sobel_mean}
    }
    \caption[Global SOBEL+MEAN Pipelines and Results]{
        Global SOBEL+MEAN Pipelines and Results
    }
    \label{fig:pipeline_global_luma_sobel_mean}
\end{figure}

\subsection{Local Features}

\paragraph{LUMA+PMEAN}

foo

\paragraph{LUMA+PMEAN2}

foo

\paragraph{CANNY+PMEAN}

foo

\begin{figure}[h]
    \centering
    \subfloat[CANNY+PMEAN+$L_2$ Pipeline]{%
        \begin{tikzpicture}[font=\tiny]
    \matrix[node grid] {
        \node [document node] (dbimg) {$I_{db}$}; &
        \node [operation node] (dbluma) {LUMA}; &
        \node [operation node] (dbcanny) {CANNY};  &
        \node [operation node] (dbcurvelet) {FDCT}; &
        \node [operation node] (dbpmean) {PMEAN}; &&
        \node [operation node] (dbquant) {VQ}; \\
        &&&&
        \node [operation node] (cluster) {CLUSTER}; &&&
        \node [operation node] (dist) {L2}; &
        \node [document node] (result) {distances}; \\
        \node [document node] (qimg) {$I_q$}; &
        \node [operation node] (qluma) {LUMA}; &&
        \node [operation node] (qcurvelet) {FDCT}; &
        \node [operation node] (qpmean) {PMEAN}; &&
        \node [operation node] (qquant) {VQ}; \\
    };
    { [start chain=going right, every join/.style={connector}]
        \chainin (dbimg);
        \chainin (dbluma) [join];
        \chainin (dbcanny) [join];
        \chainin (dbcurvelet) [join];
        \chainin (dbpmean) [join];
        { [start branch]
            \chainin (cluster) [join];
            { [start branch]
                \chainin (qquant) [join=with cluster.-2 by vh connector];
            }
            \chainin (dbquant) [join=with cluster.2 by vh connector];
        }
        \chainin (dbquant) [join];
        \chainin (dist) [join=by vh connector];
    }
    { [start chain=going right, every join/.style={connector}]
        \chainin (qimg);
        \chainin (qluma) [join];
        \chainin (qcurvelet) [join];
        \chainin (qpmean) [join];
        \chainin (qquant) [join];
        \chainin (dist) [join=by vh connector];
        \chainin (result) [join];
    }
\end{tikzpicture}

        \label{fig:pipeline_local_luma_canny_pmean_l2}
    }
    \quad
    \subfloat[CANNY+PMEAN+HI(B) Pipeline]{%
        \input{illustrations/pipelines/local_luma_canny_pmean_hi.tex}
        \label{fig:pipeline_local_luma_canny_pmean_hi}
    }
    \quad
    \subfloat[CANNY+PMEAN Results]{%
        \pgfplotstableread[]{results/l_luma_canny_results_1.csv}\resultslumacannypmean
\begin{tikzpicture}
    \begin{axis}[
        hbarplot,
        yticklabels from table={\resultslumacannypmean}{Description},
        width=0.8\textwidth,
        height=0.2\textheight,
        ]
        \addplot table[x=MeanCorrelation, y expr=\coordindex] {\resultslumacannypmean};
    \end{axis}
\end{tikzpicture}

        \label{fig:results_local_luma_canny_pmean}
    }
    \caption[Local CANNY+PMEAN Pipelines and Results]{
        Local CANNY+PMEAN Pipelines and Results
    }
    \label{fig:pipeline_local_luma_canny_pmean}
\end{figure}

\paragraph{CANNY+PMEAN2}

foo
