\section{Variants and Results}\label{sec:results}

\subsection{Global Features}

\paragraph{LUMA+MEAN}

The most straightforward combination of processing steps consists of a LUMA
input image, on which the means of $G \times G$ grid cells is computed for each
scale and angle (Figure \ref{fig:pipeline_global_luma_mean_l2} and
\ref{fig:pipeline_global_luma_mean_cos}). The distance measure COS outperforms
the $L_2$ measure (Figure \ref{fig:results_global_luma_mean}). A possible
explanation would be that the COS measure normalizes the magnitude of the
feature vector as discussed in \ref{sec:anatomy_ranking_distance_euclidean}.

\begin{figure}[h]
    \centering
    \subfloat[LUMA+MEAN+$L_2$ Pipeline]{%
        \begin{tikzpicture}[font=\tiny]
    \matrix[node grid] {
        \node [document node] (dbimg) {$I_{db}$}; &
        \node [operation node] (dbluma) {LUMA}; &
        \node [operation node] (dbcurvelet) {FDCT}; &
        \node [operation node] (dbmean) {MEAN}; \\
        \node [document node] (qimg) {$I_q$}; &
        \node [operation node] (qluma) {LUMA}; &
        \node [operation node] (qcurvelet) {FDCT}; &
        \node [operation node] (qmean) {MEAN}; &
        \node [operation node] (dist) {$L_2$}; &
        \node [document node] (result) {distances}; \\
    };
    { [start chain=going right, every join/.style={connector}]
        \chainin (dbimg);
        \chainin (dbluma) [join];
        \chainin (dbcurvelet) [join];
        \chainin (dbmean) [join];
        \chainin (dist) [join=by hv connector];
    }
    { [start chain=going right, every join/.style={connector}]
        \chainin (qimg);
        \chainin (qluma) [join];
        \chainin (qcurvelet) [join];
        \chainin (qmean) [join];
        \chainin (dist) [join];
        \chainin (result) [join];
    }
\end{tikzpicture}

        \label{fig:pipeline_global_luma_mean_l2}
    }
    \quad
    \subfloat[LUMA+MEAN+COS Pipeline]{%
        \begin{tikzpicture}[font=\tiny]
    \matrix[node grid] {
        \node [document node] (dbimg) {$I_{db}$}; &
        \node [operation node] (dbluma) {LUMA}; &
        \node [operation node] (dbcurvelet) {FDCT}; &
        \node [operation node] (dbmean) {MEAN}; \\
        \node [document node] (qimg) {$I_q$}; &
        \node [operation node] (qluma) {LUMA}; &
        \node [operation node] (qcurvelet) {FDCT}; &
        \node [operation node] (qmean) {MEAN}; &
        \node [operation node] (dist) {COS}; &
        \node [document node] (result) {distances}; \\
    };
    { [start chain=going right, every join/.style={connector}]
        \chainin (dbimg);
        \chainin (dbluma) [join];
        \chainin (dbcurvelet) [join];
        \chainin (dbmean) [join];
        \chainin (dist) [join=by hv connector];
    }
    { [start chain=going right, every join/.style={connector}]
        \chainin (qimg);
        \chainin (qluma) [join];
        \chainin (qcurvelet) [join];
        \chainin (qmean) [join];
        \chainin (dist) [join];
        \chainin (result) [join];
    }
\end{tikzpicture}

        \label{fig:pipeline_global_luma_mean_cos}
    }
    \quad
    \subfloat[LUMA+MEAN Results]{%
        \pgfplotstableread[]{results/g_luma_mean_1.csv}\resultsglumamean

\plottablexbars{scales,angles,gridsize,metric}{\resultsglumamean}

%\newcommand{\plotxbars}[#1]{%
%\begin{tikzpicture} %[trim axis left,trim axis right]
    %\begin{axis}[
        %hbarplot,
        %width=6cm,
        %%xcomb,
        %%small,
        %%y=-\baselineskip,
        %%scale only axis,
        %%enlarge y limits={abs=0.45},
        %%ytick=\empty,
        %%axis x line*=bottom,
        %%axis y line*=left,
        %%yticklabels from table={\resultsglumamean}{Description},
        %]
        %\addplot table[x=MeanCorrelation, y expr=\coordindex] {#1};
        %\addreferenceplot;
    %\end{axis}
%\end{tikzpicture}%
%}

%\newcommand{\tablexbars}[1]{
    %\pgfplotstabletypeset[columns={scales,angles,grid_size,metric,graph},
      %% Booktabs rules
      %every head row/.style={after row=\midrule},
      %every last row/.style={after row=[3ex]},
      %% Set header name
      %columns/scales/.style={string type,column name=$N_s$},
      %columns/angles/.style={string type,column name=$N_{\theta}$},
      %columns/grid_size/.style={string type,column name=$G$},
      %columns/canny_sigma/.style={string type,column name=$\sigma$},
      %columns/metric/.style={string type,column name=Metric},
      %columns/graph/.style={
        %column name={},
        %assign cell content/.code={% use \multirow for Z column:
        %\ifnum\pgfplotstablerow=0
        %\pgfkeyssetvalue{/pgfplots/table/@cell content}
        %{\multirow{\pgfplotstablegetrowsof{\resultsglumamean}\pgfplotsretval}{5cm}{\theplot}}%
        %\else
        %\pgfkeyssetvalue{/pgfplots/table/@cell content}{}%
        %\fi
        %}
      %},
    %]{\resultsglumamean}
%}

        \label{fig:results_global_luma_mean}
    }
    \caption[Global LUMA+MEAN Pipelines and Results]{
        Global LUMA+MEAN Pipelines and Results
    }
    \label{fig:pipeline_global_luma_mean}
\end{figure}

\paragraph{SOBEL+MEAN}

foo

\begin{figure}[h]
    \centering
    \subfloat[SOBEL+MEAN+$L_2$ Pipeline]{%
        \begin{tikzpicture}[font=\tiny]
    \matrix[node grid] {
        \node [document node] (dbimg) {$I_{db}$}; &
        \node [operation node] (dbluma) {LUMA}; &
        \node [operation node] (dbcanny) {CANNY};  &
        \node [operation node] (dbcurvelet) {FDCT}; &
        \node [operation node] (dbmean) {MEAN}; \\
        \node [document node] (qimg) {$I_q$}; &
        \node [operation node] (qluma) {LUMA}; &&
        \node [operation node] (qcurvelet) {FDCT}; &
        \node [operation node] (qmean) {MEAN}; &
        \node [operation node] (dist) {$L_2$}; &
        \node [document node] (result) {distances}; \\
    };
    { [start chain=going right, every join/.style={connector}]
        \chainin (dbimg);
        \chainin (dbluma) [join];
        \chainin (dbcanny) [join];
        \chainin (dbcurvelet) [join];
        \chainin (dbmean) [join];
        \chainin (dist) [join=by hv connector];
    }
    { [start chain=going right, every join/.style={connector}]
        \chainin (qimg);
        \chainin (qluma) [join];
        \chainin (qcurvelet) [join];
        \chainin (qmean) [join];
        \chainin (dist) [join];
        \chainin (result) [join];
    }
\end{tikzpicture}

        \label{fig:pipeline_global_luma_sobel_mean_l2}
    }
    \quad
    \subfloat[SOBEL+MEAN+COS Pipeline]{%
        \begin{tikzpicture}[font=\tiny]
    \matrix[node grid] {
        \node [document node] (dbimg) {$I_{db}$}; &
        \node [operation node] (dbluma) {LUMA}; &
        \node [operation node] (dbcanny) {SOBEL};  &
        \node [operation node] (dbcurvelet) {FDCT}; &
        \node [operation node] (dbmean) {MEAN}; \\
        \node [document node] (qimg) {$I_q$}; &
        \node [operation node] (qluma) {LUMA}; &&
        \node [operation node] (qcurvelet) {FDCT}; &
        \node [operation node] (qmean) {MEAN}; &
        \node [operation node] (dist) {COS}; &
        \node [document node] (result) {distances}; \\
    };
    { [start chain=going right, every join/.style={connector}]
        \chainin (dbimg);
        \chainin (dbluma) [join];
        \chainin (dbcanny) [join];
        \chainin (dbcurvelet) [join];
        \chainin (dbmean) [join];
        \chainin (dist) [join=by hv connector];
    }
    { [start chain=going right, every join/.style={connector}]
        \chainin (qimg);
        \chainin (qluma) [join];
        \chainin (qcurvelet) [join];
        \chainin (qmean) [join];
        \chainin (dist) [join];
        \chainin (result) [join];
    }
\end{tikzpicture}

        \label{fig:pipeline_global_luma_sobel_mean_cos}
    }
    \quad
    \subfloat[SOBEL+MEAN Results]{%
        \pgfplotstableread[]{results/g_luma_sobel_mean_1.csv}\resultsglumasobelmean
\plottablexbars{scales,angles,gridsize,metric}{\resultsglumasobelmean}
%\begin{tikzpicture}
    %\begin{axis}[
        %hbarplot,
        %yticklabels from table={\resultsglumasobelmean}{Description},
        %]
        %\addplot table[x=MeanCorrelation, y expr=\coordindex] {\resultsglumasobelmean};
        %\addreferenceplot;
    %\end{axis}
%\end{tikzpicture}

        \label{fig:results_global_luma_sobel_mean}
    }
    \caption[Global SOBEL+MEAN Pipelines and Results]{
        Global SOBEL+MEAN Pipelines and Results
    }
    \label{fig:pipeline_global_luma_sobel_mean}
\end{figure}

\paragraph{CANNY+MEAN}

foo

\begin{figure}[h]
    \centering
    \subfloat[CANNY+MEAN+$L_2$ Pipeline]{%
        \begin{tikzpicture}[font=\tiny]
    \matrix[node grid] {
        \node [document node] (dbimg) {$I_{db}$}; &
        \node [operation node] (dbluma) {LUMA}; &
        \node [operation node] (dbcanny) {CANNY};  &
        \node [operation node] (dbcurvelet) {FDCT}; &
        \node [operation node] (dbmean) {MEAN}; \\
        \node [document node] (qimg) {$I_q$}; &
        \node [operation node] (qluma) {LUMA}; &&
        \node [operation node] (qcurvelet) {FDCT}; &
        \node [operation node] (qmean) {MEAN}; &
        \node [operation node] (dist) {$L_2$}; &
        \node [document node] (result) {distances}; \\
    };
    { [start chain=going right, every join/.style={connector}]
        \chainin (dbimg);
        \chainin (dbluma) [join];
        \chainin (dbcanny) [join];
        \chainin (dbcurvelet) [join];
        \chainin (dbmean) [join];
        \chainin (dist) [join=by hv connector];
    }
    { [start chain=going right, every join/.style={connector}]
        \chainin (qimg);
        \chainin (qluma) [join];
        \chainin (qcurvelet) [join];
        \chainin (qmean) [join];
        \chainin (dist) [join];
        \chainin (result) [join];
    }
\end{tikzpicture}

        \label{fig:pipeline_global_luma_canny_mean_l2}
    }
    \quad
    \subfloat[CANNY+MEAN+COS Pipeline]{%
        \begin{tikzpicture}[font=\tiny]
    \matrix[node grid] {
        \node [document node] (dbimg) {$I_{db}$}; &
        \node [operation node] (dbluma) {LUMA}; &
        \node [operation node] (dbcanny) {CANNY};  &
        \node [operation node] (dbcurvelet) {FDCT}; &
        \node [operation node] (dbmean) {MEAN}; \\
        \node [document node] (qimg) {$I_q$}; &
        \node [operation node] (qluma) {LUMA}; &&
        \node [operation node] (qcurvelet) {FDCT}; &
        \node [operation node] (qmean) {MEAN}; &
        \node [operation node] (dist) {COS}; &
        \node [document node] (result) {distances}; \\
    };
    { [start chain=going right, every join/.style={connector}]
        \chainin (dbimg);
        \chainin (dbluma) [join];
        \chainin (dbcanny) [join];
        \chainin (dbcurvelet) [join];
        \chainin (dbmean) [join];
        \chainin (dist) [join=by hv connector];
    }
    { [start chain=going right, every join/.style={connector}]
        \chainin (qimg);
        \chainin (qluma) [join];
        \chainin (qcurvelet) [join];
        \chainin (qmean) [join];
        \chainin (dist) [join];
        \chainin (result) [join];
    }
\end{tikzpicture}

        \label{fig:pipeline_global_luma_canny_mean_cos}
    }
    \quad
    \subfloat[CANNY+MEAN Results]{%
        \pgfplotstableread[]{results/g_luma_canny_mean_1.csv}\resultsglumacannymean
\begin{tikzpicture}
    \begin{axis}[
        hbarplot,
        yticklabels from table={\resultsglumacannymean}{Description},
        ]
        \addplot table[x=MeanCorrelation, y expr=\coordindex] {\resultsglumacannymean};
        \addreferenceplot;
    \end{axis}
\end{tikzpicture}

        \label{fig:results_global_luma_canny_mean}
    }
    \caption[Global CANNY+MEAN Pipelines and Results]{
        Global CANNY+MEAN Pipelines and Results
    }
    \label{fig:pipeline_global_luma_canny_mean}
\end{figure}

\subsection{Local Features}

\paragraph{CANNY+PMEAN}

\begin{figure}[h]
    \centering
    \subfloat[CANNY+PMEAN+$L_2$ Pipeline]{%
        \begin{tikzpicture}[font=\tiny]
    \matrix[node grid] {
        \node [document node] (dbimg) {$I_{db}$}; &
        \node [operation node] (dbluma) {LUMA}; &
        \node [operation node] (dbcanny) {CANNY};  &
        \node [operation node] (dbcurvelet) {FDCT}; &
        \node [operation node] (dbpmean) {PMEAN}; &&
        \node [operation node] (dbquant) {VQ}; \\
        &&&&
        \node [operation node] (cluster) {CLUSTER}; &&&
        \node [operation node] (dist) {L2}; &
        \node [document node] (result) {distances}; \\
        \node [document node] (qimg) {$I_q$}; &
        \node [operation node] (qluma) {LUMA}; &&
        \node [operation node] (qcurvelet) {FDCT}; &
        \node [operation node] (qpmean) {PMEAN}; &&
        \node [operation node] (qquant) {VQ}; \\
    };
    { [start chain=going right, every join/.style={connector}]
        \chainin (dbimg);
        \chainin (dbluma) [join];
        \chainin (dbcanny) [join];
        \chainin (dbcurvelet) [join];
        \chainin (dbpmean) [join];
        { [start branch]
            \chainin (cluster) [join];
            { [start branch]
                \chainin (qquant) [join=with cluster.-2 by vh connector];
            }
            \chainin (dbquant) [join=with cluster.2 by vh connector];
        }
        \chainin (dbquant) [join];
        \chainin (dist) [join=by vh connector];
    }
    { [start chain=going right, every join/.style={connector}]
        \chainin (qimg);
        \chainin (qluma) [join];
        \chainin (qcurvelet) [join];
        \chainin (qpmean) [join];
        \chainin (qquant) [join];
        \chainin (dist) [join=by vh connector];
        \chainin (result) [join];
    }
\end{tikzpicture}

        \label{fig:pipeline_local_luma_canny_pmean_l2}
    }
    \quad
    \subfloat[CANNY+PMEAN+HI(B) Pipeline]{%
        \begin{tikzpicture}[font=\tiny]
    \matrix[node grid] {
        \node [document node] (dbimg) {$I_{db}$}; &
        \node [operation node] (dbluma) {LUMA}; &
        \node [operation node] (dbcanny) {CANNY};  &
        \node [operation node] (dbcurvelet) {FDCT}; &
        \node [operation node] (dbpmean) {PMEAN}; &&
        \node [operation node] (dbquant) {VQ}; \\
        &&&&
        \node [operation node] (cluster) {CLUSTER}; &&&
        \node [operation node] (dist) {$HI(B)$}; &
        \node [document node] (result) {distances}; \\
        \node [document node] (qimg) {$I_q$}; &
        \node [operation node] (qluma) {LUMA}; &&
        \node [operation node] (qcurvelet) {FDCT}; &
        \node [operation node] (qpmean) {PMEAN}; &&
        \node [operation node] (qquant) {VQ}; \\
    };
    { [start chain=going right, every join/.style={connector}]
        \chainin (dbimg);
        \chainin (dbluma) [join];
        \chainin (dbcanny) [join];
        \chainin (dbcurvelet) [join];
        \chainin (dbpmean) [join];
        { [start branch]
            \chainin (cluster) [join];
            { [start branch]
                \chainin (qquant) [join=with cluster.-2 by hv connector];
            }
            \chainin (dbquant) [join=with cluster.2 by hv connector];
        }
        \chainin (dbquant) [join];
        \chainin (dist) [join=by hv connector];
    }
    { [start chain=going right, every join/.style={connector}]
        \chainin (qimg);
        \chainin (qluma) [join];
        \chainin (qcurvelet) [join];
        \chainin (qpmean) [join];
        \chainin (qquant) [join];
        \chainin (dist) [join=by hv connector];
        \chainin (result) [join];
    }
\end{tikzpicture}

        \label{fig:pipeline_local_luma_canny_pmean_hi}
    }
    \quad
    \subfloat[CANNY+PMEAN Results]{%
        \pgfplotstableread[]{results/l_luma_canny_pmean_1.csv}\resultsllumacannypmean
\begin{tikzpicture}
    \begin{axis}[
        hbarplot,
        yticklabels from table={\resultsllumacannypmean}{Description},
        %width=0.8\textwidth,
        %height=0.2\textheight,
        ]
        \addplot table[x=MeanCorrelation, y expr=\coordindex] {\resultsllumacannypmean};
        \addreferenceplot;
    \end{axis}
\end{tikzpicture}

        \label{fig:results_local_luma_canny_pmean}
    }
    \caption[Local CANNY+PMEAN Pipelines and Results]{
        Local CANNY+PMEAN Pipelines and Results
    }
    \label{fig:pipeline_local_luma_canny_pmean}
\end{figure}

\paragraph{CANNY+PMEAN2}
