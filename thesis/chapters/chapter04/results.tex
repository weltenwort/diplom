\section{Variants and Results}\label{sec:results}

Some of the variations below include a Canny edge detector as a preprocessing
step. The parameter $\sigma$ determines the size of the gaussian kernel used by
the Canny algorithm. A value of $\sigma = 1.5$ was determined to be appropriate
for the image dataset used in the benchmark. Value larger than $\sigma = 2$
tend not to detect any edges, while smaller values produced more "false" edges
resulting from noise in the images.

The SEGMENT preprocessing step could unfortunately not use the same $1024
\times 768$ pixel images as inputs as the other preprocessors, because the
implementation provided by Arbelaez et al.\ \autocite{arbelaez_contour_2011}
could not be run on any machine available due to extreme memory requirements.
Therefore, the images in the database are rescaled to $320 \times 240$ pixels
for the evaluation of this algorithm. This might lead to loss of small features
    and puts limits on some parameters like the grid size $G$.

All pipeline variants utilize the FDCT to extract curve information. The number
of scales and the number of angles at the coarsest scale will be called $N_s$
and $N_{\theta}$ respectively. Based on experimentation and other publications
using the curvelet transform \autocite{mandal_curvelet_2009}
\autocite{guha_curvelet_????}, 4 scales and 12 angles are used in almost all
cases, since larger values did not consistently lead to better results. The
size of the codebook is set at 1000 visual words, as values beyond that did
not show improvements in evaluations performed by Nowak at al.\
\autocite{nowak_sampling_2006}. Eitz et al.\ \autocite{eitz_sketch-based_2010}
reported similar optimal values of 500 to 1000 visual words and attributed the
low number to the sparsity of edge-like features in sketches.

Throughout the correlation coefficient graphs in the following sections, the
red line denotes the best result obtained by Eitz et al.\ in
\autocite{eitz_sketch-based_2010} using the SHOG descriptor.

\subsection{Global Features}

The processing steps applied to the database images and the query images in the
pipelines based on global features are almost identical, with the exception of
the CANNY, SOBEL and SEGMENT steps for the database images. The curvelet
responses are averaged on a grid and the features are ranked using the $L_2$
and the COS distance measures.

\paragraph{LUMA+MEAN}

The most straightforward combination of processing steps consists of a LUMA
input image, on which the means of $G \times G$ grid cells is computed for each
scale and angle (\autoref{fig:pipeline_global_luma_mean}).  Varying $G$
determines the feature size that can be encoded best. Setting $G=12$ seems to
yield the best correlation, although the advantage over $G=8$ and $G=16$ is
below $0.01$. The distance measure COS outperforms the $L_2$ measure
(\autoref{tab:results_global_luma_mean}). A possible explanation would be that
the COS measure normalizes the magnitude of the feature vector as discussed in
\autoref{sec:anatomy_ranking_distance_euclidean}.

\begin{figure}[h]
    \centering
    \input{illustrations/pipelines/global_luma_mean.tex}
    \caption[Global LUMA+MEAN Pipelines]{
        Global LUMA+MEAN Pipelines
    }
    \label{fig:pipeline_global_luma_mean}
\end{figure}

\begin{table}[h]
    \centering
    \pgfplotstableread[]{results/g_luma_mean_1.csv}\resultsglumamean
\begin{tikzpicture}
    \begin{axis}[
        hbarplot,
        yticklabels from table={\resultsglumamean}{Description},
        ]
        \addplot table[x=MeanCorrelation, y expr=\coordindex] {\resultsglumamean};
        \addreferenceplot;
    \end{axis}
\end{tikzpicture}

    \caption[Global LUMA+MEAN Results]{
        Global LUMA+MEAN Results
    }
    \label{tab:results_global_luma_mean}
\end{table}

\FloatBarrier
\paragraph{CANNY+MEAN}

In addition to reading the images like in the previous LUMA+MEAN configuration,
this pipeline applies a CANNY processing step to the database images in an
attempt to bring the query and database image domains closer together
(\autoref{fig:pipeline_global_luma_canny_mean}). Again, the COS distance
measure produces the best rankings
(\autoref{tab:results_global_luma_canny_mean}).  Surprisingly, the Canny edge
detector does not lead to increased performance in comparison to plain the LUMA
preprocessing step.

\begin{figure}[h]
    \centering
    \begin{tikzpicture}[font=\tiny]
    \matrix[node grid] {
        \node [document node] (dbimg) {$I_{db}$}; &
        \node [operation node] (dbluma) {LUMA}; &
        \node [operation node] (dbcanny) {CANNY};  &
        \node [operation node] (dbcurvelet) {FDCT}; &
        \node [operation node] (dbmean) {MEAN}; \\
        \node [document node] (qimg) {$I_q$}; &
        \node [operation node] (qluma) {LUMA}; &&
        \node [operation node] (qcurvelet) {FDCT}; &
        \node [operation node] (qmean) {MEAN}; \\
    };

    \node [operation node, split node=2, right=3ex of $(dbmean.east)!0.5!(qmean.east)$] (dist) {$L_2$ \nodepart{two} COS};
    \node [document node, right=of dist] (result) {distances};

    \node [parameter node, above=of dbcanny] (dbcannyparam) {$\sigma$};
    \node [parameter node, above=of dbcurvelet] (dbcurveletparam) {$(N_s, N_{\theta})$};
    \node [parameter node, above=of dbmean] (dbmeanparam) {$G$};
    \node [parameter node, below=of qcurvelet] (qcurveletparam) {$(N_s, N_{\theta})$};
    \node [parameter node, below=of qmean] (qmeanparam) {$G$};

    \path [parameter connector] (dbcannyparam) -- (dbcanny);
    \path [parameter connector] (dbcurveletparam) -- (dbcurvelet);
    \path [parameter connector] (dbmeanparam) -- (dbmean);
    \path [parameter connector] (qcurveletparam) -- (qcurvelet);
    \path [parameter connector] (qmeanparam) -- (qmean);

    { [start chain=going right, every join/.style={connector}]
        \chainin (dbimg);
        \chainin (dbluma) [join];
        \chainin (dbcanny) [join];
        \chainin (dbcurvelet) [join];
        \chainin (dbmean) [join];
        \chainin (dist) [join=with dbmean.east by hvh connector top];
    }
    { [start chain=going right, every join/.style={connector}]
        \chainin (qimg);
        \chainin (qluma) [join];
        \chainin (qcurvelet) [join];
        \chainin (qmean) [join];
        \chainin (dist) [join=with qmean.east by hvh connector bottom];
        \chainin (result) [join];
    }
\end{tikzpicture}

    \caption[Global CANNY+MEAN Pipelines]{
        Global CANNY+MEAN Pipelines
    }
    \label{fig:pipeline_global_luma_canny_mean}
\end{figure}

\begin{table}[h]
    \centering
    \pgfplotstableread[]{results/g_luma_canny_mean_1.csv}\resultsglumacannymean
\plottablexbars{scales,angles,gridsize,cannysigma,metric}{\resultsglumacannymean}

    \caption[Global CANNY+MEAN Results]{
        Global CANNY+MEAN Results
    }
    \label{tab:results_global_luma_canny_mean}
\end{table}

\FloatBarrier
\paragraph{SOBEL+MEAN}

The SOBEL step used in this variant also attempts to bring the database images
into the sketch domain (\autoref{fig:pipeline_global_luma_sobel_mean}). The
results are slightly better than with the CANNY preprocessor
(\autoref{tab:results_global_luma_sobel_mean}).

\begin{figure}[h!]
    \centering
    \input{illustrations/pipelines/global_luma_sobel_mean.tex}
    \caption[Global SOBEL+MEAN Pipelines]{
        Global SOBEL+MEAN Pipelines
    }
    \label{fig:pipeline_global_luma_sobel_mean}
\end{figure}

\begin{table}[h!]
    \centering
    \pgfplotstableread[]{results/g_luma_sobel_mean_1.csv}\resultsglumasobelmean
\begin{tikzpicture}
    \begin{axis}[
        hbarplot,
        yticklabels from table={\resultsglumasobelmean}{Description},
        ]
        \addplot table[x=MeanCorrelation, y expr=\coordindex] {\resultsglumasobelmean};
        \addreferenceplot;
    \end{axis}
\end{tikzpicture}

    \caption[Global SOBEL+MEAN Results]{
        Global SOBEL+MEAN Results
    }
    \label{tab:results_global_luma_sobel_mean}
\end{table}

\FloatBarrier
\paragraph{SEGMENT+MEAN}

With the gPb contour detector in the SEGMENT step to find edges in the database
images (\autoref{fig:pipeline_global_luma_segment_mean}), the $L_2$ distance
metric produces results comparable to the COS metric in the CANNY+MEAN variant
(\autoref{tab:results_global_luma_segment_mean}).  Unlike in the other cases,
the COS distance measure performs worse than the $L_2$ metric.

\begin{figure}[h]
    \centering
    \begin{tikzpicture}[font=\tiny]
    \matrix[node grid] {
        \node [document node] (dbimg) {$I_{db}$}; &
        \node [operation node] (dbsegment) {SEGMENT}; &
        \node [operation node] (dbcurvelet) {FDCT}; &
        \node [operation node] (dbmean) {MEAN}; \\
        \node [document node] (qimg) {$I_q$}; &
        \node [operation node] (qluma) {LUMA}; &
        \node [operation node] (qcurvelet) {FDCT}; &
        \node [operation node] (qmean) {MEAN}; \\
    };

    \node [operation node, split node=2, right=3ex of $(dbmean.east)!0.5!(qmean.east)$] (dist) {$L_2$ \nodepart{two} COS};
    \node [document node, right=of dist] (result) {distances};

    { [start chain=going right, every join/.style={connector}]
        \chainin (dbimg);
        \chainin (dbsegment) [join];
        \chainin (dbcurvelet) [join];
        \chainin (dbmean) [join];
        \chainin (dist) [join=with dbmean.east by hvh connector top];
    }
    { [start chain=going right, every join/.style={connector}]
        \chainin (qimg);
        \chainin (qluma) [join];
        \chainin (qcurvelet) [join];
        \chainin (qmean) [join];
        \chainin (dist) [join=with qmean.east by hvh connector bottom];
        \chainin (result) [join];
    }
\end{tikzpicture}

    \caption[Global SEGMENT+MEAN Pipelines]{
        Global SEGMENT+MEAN Pipelines
    }
    \label{fig:pipeline_global_luma_segment_mean}
\end{figure}

\begin{table}[h]
    \centering
    \pgfplotstableread[]{results/g_luma_segment_mean_1.csv}\resultsglumasegmentmean
\plottablexbars{scales,angles,gridsize,metric}{\resultsglumasegmentmean}

    \caption[Global SEGMENT+MEAN Results]{
        Global SEGMENT+MEAN Results
    }
    \label{tab:results_global_luma_segment_mean}
\end{table}

\FloatBarrier

\subsection{Local Features}

\paragraph{LUMA+PMEAN}

foo

\begin{figure}[h]
    \centering
    \subfloat[LUMA+PMEAN+$L_2$ Pipeline]{%
        \begin{tikzpicture}[font=\tiny]
    \matrix[node grid] {
        \node [document node] (dbimg) {$I_{db}$}; &
        \node [operation node] (dbluma) {LUMA}; &
        \node [operation node] (dbcurvelet) {FDCT}; &
        \node [operation node] (dbpmean) {PMEAN}; &&
        \node [operation node] (dbquant) {VQ}; \\
        &&&
        \node [operation node] (cluster) {CLUSTER}; \\
        \node [document node] (qimg) {$I_q$}; &
        \node [operation node] (qluma) {LUMA}; &
        \node [operation node] (qcurvelet) {FDCT}; &
        \node [operation node] (qpmean) {PMEAN}; &&
        \node [operation node] (qquant) {VQ}; \\
    };
    \node [operation node, split node=3, right=3ex of $(dbquant.east)!0.5!(qquant.east)$] (dist) {$L_2$ \nodepart{two} HI(B) \nodepart{three} EMD};
    \node [document node, right=of dist] (result) {distances};

    { [start chain=going right, every join/.style={connector}]
        \chainin (dbimg);
        \chainin (dbluma) [join];
        \chainin (dbcurvelet) [join];
        \chainin (dbpmean) [join];
        { [start branch]
            \chainin (cluster) [join];
            { [start branch]
                \chainin (qquant) [join=with cluster.-2 by hv connector];
            }
            \chainin (dbquant) [join=with cluster.2 by hv connector];
        }
        \chainin (dbquant) [join];
        \chainin (dist) [join=with dbquant.east by hvh connector top];
    }
    { [start chain=going right, every join/.style={connector}]
        \chainin (qimg);
        \chainin (qluma) [join];
        \chainin (qcurvelet) [join];
        \chainin (qpmean) [join];
        \chainin (qquant) [join];
        \chainin (dist) [join=with qquant.east by hvh connector bottom];
        \chainin (result) [join];
    }
\end{tikzpicture}

        \label{fig:pipeline_local_luma_pmean_l2}
    }
    \quad
    \subfloat[LUMA+PMEAN+HI(B) Pipeline]{%
        \input{illustrations/pipelines/local_luma_pmean_hi.tex}
        \label{fig:pipeline_local_luma_pmean_hi}
    }
    \caption[Local LUMA+PMEAN Pipelines]{
        Local LUMA+PMEAN Pipelines
    }
    \label{fig:pipeline_local_luma_pmean}
\end{figure}

\begin{table}[h]
    \pgfplotstableread[]{results/l_luma_pmean_1.csv}\resultsllumapmean
\plottablexbars{scales,angles,gridsize,patchsize,metric}{\resultsllumapmean}
%\begin{tikzpicture}
    %\begin{axis}[
        %hbarplot,
        %yticklabels from table={\resultsllumacannypmean}{Description},
        %%width=0.8\textwidth,
        %%height=0.2\textheight,
        %]
        %\addplot table[x=MeanCorrelation, y expr=\coordindex] {\resultsllumacannypmean};
        %\addreferenceplot;
    %\end{axis}
%\end{tikzpicture}

    \caption[Local LUMA+PMEAN Results]{
        Local LUMA+PMEAN Results
    }
    \label{tab:results_local_luma_pmean}
\end{table}

\paragraph{LUMA+PMEAN2}

foo

\paragraph{CANNY+PMEAN}

foo

\begin{figure}[h]
    \centering
    \subfloat[CANNY+PMEAN+$L_2$ Pipeline]{%
        \begin{tikzpicture}[font=\tiny]
    \matrix[node grid] {
        \node [document node] (dbimg) {$I_{db}$}; &
        \node [operation node] (dbluma) {LUMA}; &
        \node [operation node] (dbcanny) {CANNY};  &
        \node [operation node] (dbcurvelet) {FDCT}; &
        \node [operation node] (dbpmean) {PMEAN}; &&
        \node [operation node] (dbquant) {VQ}; \\
        &&&&
        \node [operation node] (cluster) {CLUSTER}; &&&
        \node [operation node] (dist) {L2}; &
        \node [document node] (result) {distances}; \\
        \node [document node] (qimg) {$I_q$}; &
        \node [operation node] (qluma) {LUMA}; &&
        \node [operation node] (qcurvelet) {FDCT}; &
        \node [operation node] (qpmean) {PMEAN}; &&
        \node [operation node] (qquant) {VQ}; \\
    };
    { [start chain=going right, every join/.style={connector}]
        \chainin (dbimg);
        \chainin (dbluma) [join];
        \chainin (dbcanny) [join];
        \chainin (dbcurvelet) [join];
        \chainin (dbpmean) [join];
        { [start branch]
            \chainin (cluster) [join];
            { [start branch]
                \chainin (qquant) [join=with cluster.-2 by vh connector];
            }
            \chainin (dbquant) [join=with cluster.2 by vh connector];
        }
        \chainin (dbquant) [join];
        \chainin (dist) [join=by vh connector];
    }
    { [start chain=going right, every join/.style={connector}]
        \chainin (qimg);
        \chainin (qluma) [join];
        \chainin (qcurvelet) [join];
        \chainin (qpmean) [join];
        \chainin (qquant) [join];
        \chainin (dist) [join=by vh connector];
        \chainin (result) [join];
    }
\end{tikzpicture}

        \label{fig:pipeline_local_luma_canny_pmean_l2}
    }
    \quad
    \subfloat[CANNY+PMEAN+HI(B) Pipeline]{%
        \input{illustrations/pipelines/local_luma_canny_pmean_hi.tex}
        \label{fig:pipeline_local_luma_canny_pmean_hi}
    }
    %\quad
    %\subfloat[CANNY+PMEAN Results]{%
        %\pgfplotstableread[]{results/l_luma_canny_results_1.csv}\resultslumacannypmean
\begin{tikzpicture}
    \begin{axis}[
        hbarplot,
        yticklabels from table={\resultslumacannypmean}{Description},
        width=0.8\textwidth,
        height=0.2\textheight,
        ]
        \addplot table[x=MeanCorrelation, y expr=\coordindex] {\resultslumacannypmean};
    \end{axis}
\end{tikzpicture}

        %\label{fig:results_local_luma_canny_pmean}
    %}
    \caption[Local CANNY+PMEAN Pipelines]{
        Local CANNY+PMEAN Pipelines
    }
    \label{fig:pipeline_local_luma_canny_pmean}
\end{figure}

\begin{table}[h]
    \pgfplotstableread[]{results/l_luma_canny_results_1.csv}\resultslumacannypmean
\begin{tikzpicture}
    \begin{axis}[
        hbarplot,
        yticklabels from table={\resultslumacannypmean}{Description},
        width=0.8\textwidth,
        height=0.2\textheight,
        ]
        \addplot table[x=MeanCorrelation, y expr=\coordindex] {\resultslumacannypmean};
    \end{axis}
\end{tikzpicture}

    \caption[Local CANNY+PMEAN Results]{
        Local CANNY+PMEAN Results
    }
    \label{tab:results_local_luma_canny_pmean}
\end{table}

\paragraph{CANNY+PMEAN2}

foo



