\chapter{Proposed Solution}\label{ch:solution}
Proposed solution goes here\dots

\section{Input Format}
\begin{itemize}
    \item Luma component (Y') of Y'UV representation
    \item Gradient magnitude of Sobel operator of luma component
    \item Canny edge map of luma component
    \item gPb
\end{itemize}

\section{Feature Extraction}
\begin{itemize}
    \item Global features: mean and standard deviation
    \item Local features: visual words via k-means clustering
    \item great comparison of sampling for k-means clustered vws [nowak06]
\end{itemize}

\section{Distance Metric}
\begin{itemize}
    \item Euclidean Distance
    \item cosine distance?
    \item EMD?
\end{itemize}

%\begin{figure}
    %\scriptsize
    %\begin{tikzpicture}[
            %node distance=1em and 1em,
            %every node/.style={font=\sf},
            %point/.style={},
            %rowHeader/.style={
                %draw=black,
                %text width=5.5em,
            %},
            %colHeader/.style={
                %draw=black,
            %},
            %block/.style={
                %rectangle,
                %draw=black!80, thick,
                %fill=black!10,
                %text width=5.5em,
                %text centered,
                %minimum height=2em,
                %anchor=north,
            %},
            %flow/.style={
                %->,
                %draw=black!40,
                %ultra thick,
            %}
        %]
        %\def\brshift{0em and 2em};

        %% column headers
        %\node[rowHeader] (hCol1) {};
        %\node[rowHeader, right=of hCol1] (hCol2) {};
        %\node[rowHeader, right=of hCol2] (hCol3) {};
        %\node[rowHeader, right=of hCol3] (hCol4) {};
        %\node[rowHeader, right=of hCol4] (hCol5) {};
        %\node[rowHeader, right=of hCol5] (hCol6) {};

        %% row headers
        %\node[colHeader, minimum height=3em, below left=1em and 0em of hCol1] (hRow1) {};
        %\node[colHeader, minimum height=3em, below=of hRow1] (hRow2) {};
        %\node[colHeader, minimum height=3em, below=of hRow2] (hRow3) {};

        %\node[block, minimum height=23em] at (hCol1 |- hRow1.north) (readImage) {Read Image};

        %\node[block] at (hCol2 |- hRow1.north) (extractLuma) {Extract Luma};
        %\node[block] at (hCol2 |- hRow2.north) (applySobel) {Apply Sobel Operator};
        %\node[block] at (hCol2 |- hRow3.north) (applyCanny) {Apply Canny Operator};

        %\node[block, minimum height=11em] at (hCol3 |- hRow1.north) (applyCurvelet) {Apply Curvelet Transform};

        %\node[block, minimum height=11em] at (hCol4 |- hRow1.north) (sample) {Determine Samples};

        %\node[block, minimum height=11em] at (hCol5 |- hRow1.north) (calculateMeans) {Calculate Means};

        %\node[block, minimum height=11em] at (hCol6 |- hRow1.north) (rankEuclidean) {Rank using Euclidean Metric};

        %\foreach \row in {1, 2, 3} {
            %\draw[flow] (hCol1.east |- hRow\row.center) -- (hCol2.west |- hRow\row.center);
            %\draw[flow] (hCol2.east |- hRow\row.center) -- (hCol3.west |- hRow\row.center);
            %\draw[flow] (hCol3.east |- hRow\row.center) -- (hCol4.west |- hRow\row.center);
            %\draw[flow] (hCol4.east |- hRow\row.center) -- (hCol5.west |- hRow\row.center);
            %\draw[flow] (hCol5.east |- hRow\row.center) -- (hCol6.west |- hRow\row.center);
        %}
        %%\def\f1{hRow1.center}
        %%\draw[flow] (\f1 -| readImage.east) -- (\f1 -| extractLuma.west);
        %%\draw[flow] (\f1 -| extractLuma.east) -- (\f1 -| applyCurvelet.west);
        %%\draw[flow] (\f1 -| applyCurvelet.east) -- (\f1 -| sample.west);
        %%\draw[flow] (\f1 -| sample.east) -- (\f1 -| calculateMeans.west);

        %%\def\f2{hRow2.center}
        %%\draw[flow] (\f2 -| readImage.east) -- (\f2 -| applySobel.west);
    %\end{tikzpicture}
%\end{figure}
