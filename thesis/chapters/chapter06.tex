\chapter{Conclusion}\label{ch:conclusion}

This thesis examined the suitability of image descriptors for sketch-based
image retrieval. In particular, the theoretical background of the Fast Discrete
Curvelet Transform was explained and its use in combination with established
retrieval system architectures was demonstrated. Querying a database of
photographs using hand-drawn sketches was used as a representative example of
cross-domain image retrieval.

The theoretical discussion in literature of the Curvelet transform's advantages
over related algorithms such as the Gabor filter indicate it might be
especially well suited for sketch-based image retrieval. To evaluate this,
common structural features of retrieval systems were examined and based on that
several processing pipeline variations were implemented, that utilize the Fast
Discrete Curvelet Transform. This included relying on global image information
as well as using local neighborhoods to extract image features. The performance
of these pipelines was measured using both a cross-domain benchmark and an
intra-domain benchmark.

The results showed, that the curvelet-based descriptors can compete with other
descriptors described in literature, although the implementations used in this
paper did not exceed the best among those. For cross-domain retrieval a local
feature descriptor, that performed edge detection on the photographs, was most
successful. The intra-domain evaluation on the other hand resulted in a global
descriptor showing slightly better performance. In both cases the advantage of
one type over the other was small and might be attributed to limitations of the
benchmark dataset. Also noticeable in both cases was a broad distribution of
the resulting values for different query images and categories, that was
consistent in several descriptors. This could be an indication, that the
semantic and sensory gaps were too large for the algorithms to overcome, for
example when the photographs or sketches contained large amounts of clutter or
a category included ambiguous representations of an object.

\section{Future Research}\label{sec:conclusion_future_research}

why are some categories so difficult?
application to specific retrieval scenarios, e.g. medical images
