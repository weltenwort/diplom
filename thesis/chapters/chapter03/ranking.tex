\section{Ranking}

As the last step of the pipeline, ranking operates on the signatures produced
by prior extraction steps. It outputs a sequence of database images, sorted by
the distance to the query image in ascending order. The distance can be
determined using various metrics and similarity measures, which have been
detailed in Section \ref{sec:anatomy_ranking_distance}. This section lists the
metrics used in the experiments and labels them for later reference.

\paragraph{$L_2$}

The simplest and most widely used distance metric calculates the euclidean
distance between the query image's signature and each database image's
signature.

\paragraph{COS}

Following the definition of the cosine distance in Section
\ref{sec:anatomy_ranking_distance}, this calculates the distance of two
signatures vectors via the angle between them.

\paragraph{HI}

When comparing histograms of features, the histogram intersection measure
$s_{HI}$ has been shown to be superior to the euclidean distance in most of the
cases \autocite{wu_beyond_2009}. It has the added benefit of allowing for
partial matches in a signature. As can be seen from the definition in Section
\ref{sec:anatomy_ranking_distance}, the result lies within $[0, 1]$ with $1$
being a perfect match. Therefore, $1 - s_{HI}$ is used as a distance value to
sort the result list.

\paragraph{HIB}

The binary variant of the histogram intersection measure converts the bin
counts into a boolean representation, where a $1$ indicates the presence of a
codeword irrespective of the actual count, and a $0$ denotes a codeword's
absence. The resulting sequence of zeros and ones is then treated like in the
HI measure.

\paragraph{EMD}

The Earth Mover's Distance is solved using a simplex algorithm variant, which
has an exponential worst case complexity. That makes it computationally more
expensive than the linear complexity measures described previously.
