\section{Signature Extraction}

Each method of signature extraction described in this section has the fast
discrete curvelet transform at its heart. The curvelet transform has two main
parameters, that influence the result: The number of angles $N_{\theta}$ used
at the coarest scale and the number of scales $N_j$, which corresponds to the
number of concentric squares shown in \ref{fig:curvelet_discrete_tilings}.
Experiments conducted to determine the optimal values of these parameters have
shown that using more scales than 4 does not provide benefits that would
justify the increased amount of processing time. Furthermore, since the
coarsest scale is non-directional, as explained in section
\ref{sec:background_cct}, it is ignored in further computations.

The response image generated by the FDCT for each pair of scale and angle is
too large to be considered for the signature directly. Therefore the response
image $C_{s, \theta}$ for scale $s$ and angle $\theta$ is subdivided into $n^2$
equally sized grid cells $G_{s, \theta, x, y}$ with $x, y \in 1, 2, \dots, n$:
\begin{equation*}
    C_{s,\theta} =
    \begin{bmatrix}
        G_{s,\theta,1,1} & G_{s,\theta,1,2} & \cdots & G_{s,\theta,1,n} \\
        G_{s,\theta,2,1} & G_{s,\theta,2,2} & \cdots & G_{s,\theta,2,n} \\
        \vdots  & \vdots  & \ddots & \vdots  \\
        G_{s,\theta,n,1} & G_{s,\theta,n,2} & \cdots & G_{s,\theta,n,n} \\
    \end{bmatrix}
\end{equation*}
For each of these grid cells, the mean $\bar{C}_{s, \theta}$ is calculated:
\begin{align*}
    \bar{C}_{s,\theta} &=
    \begin{bmatrix}
        mean(G_{s,\theta,1,1}) & mean(G_{s,\theta,1,2}) & \cdots & mean(G_{s,\theta,1,n}) \\
        mean(G_{s,\theta,2,1}) & mean(G_{s,\theta,2,2}) & \cdots & mean(G_{s,\theta,2,n}) \\
        \vdots  & \vdots  & \ddots & \vdots  \\
        mean(G_{s,\theta,n,1}) & mean(G_{s,\theta,n,2}) & \cdots & mean(G_{s,\theta,n,n}) \\
    \end{bmatrix} \\
    &=
    \begin{bmatrix}
        \bar{c}_{s,\theta,1,1} & \bar{c}_{s,\theta,1,2} & \cdots & \bar{c}_{s,\theta,1,n} \\
        \bar{c}_{s,\theta,2,1} & \bar{c}_{s,\theta,2,2} & \cdots & \bar{c}_{s,\theta,2,n} \\
        \vdots  & \vdots  & \ddots & \vdots  \\
        \bar{c}_{s,\theta,n,1} & \bar{c}_{s,\theta,n,2} & \cdots & \bar{c}_{s,\theta,n,n} \\
    \end{bmatrix}
\end{align*}

\subsection{Global Features}

\paragraph{MEAN}

The global approach to signature extraction simply takes the family of matrices
$\bar{C}_{s, \theta}$ and concatenates them as the image signature.

\subsection{Local Features}

The local feature extraction methods used here follow the bag-of-features
approach, that aims to represent an image using a set of local feature
descriptions or "visual words" similar to what was described in
\autocite{sivic_video_2003}. The exact way to extract the words will be
detailed towards the end of this section.

The set of visual words extracted from the images are diverse and hard to
compare. In order to create meaningful image signatures from these words, the
whole set is condensed into a dictionary using k-means clustering. As already
discussed in \ref{sec:anatomy_signature_extraction}, the goal thereof is to
derive a dictionary of predefined size that contains the visual words
corresponding to the most discriminating features of the images in the image
database. A universally optimal size for the codebook does not seem to exist,
as Nowak et al.\ \autocite{nowak_sampling_2006} observe an increase in accuracy
up to 1000 words, but overfitting for some sampling algorithms beyond that. At
the same time \autocite{yang_evaluating_2007} report optimal sizes of 20000 to
80000 depending on the image database. In \autocite{eitz_sketch-based_2010} a
size of 1000 visual words was found optimal for sketches.

\paragraph{PMEAN}

Continuing from the set of matrices $\bar{C}_{s, \theta}$, this algorithm
densly samples each matrix by sliding a window of size $m \times m, m < n$
across it. This results in $n - m + 1$ parts $\bar{W}_{s, \theta, u, v}$ with
$u, v \in 1, \dots, n - m + 1$:
\begin{equation*}
    \bar{W}_{s,\theta,u,v} =
    \begin{bmatrix}
        \bar{c}_{s,\theta,u,v} & \bar{c}_{s,\theta,u,v+1} & \cdots & \bar{c}_{s,\theta,u,v+m} \\
        \bar{c}_{s,\theta,u+1,v} & \bar{c}_{s,\theta,u+1,v+1} & \cdots & \bar{c}_{s,\theta,u+1,v+m} \\
        \vdots  & \vdots  & \ddots & \vdots  \\
        \bar{c}_{s,\theta,u+m,v} & \bar{c}_{s,\theta,u+m,v+1} & \cdots & \bar{c}_{s,\theta,u+m,v+m} \\
    \end{bmatrix}
\end{equation*}
For each pair $u, v$ these matrices are concatenated in a consistent way and
stored as the feature vectors of the image. That way, the algorithm derives $u
\cdot v$ vectors of length $N_s \cdot N_{\theta_s} \cdot m^2$ from each image,
where $N_{\theta_s}$ is the number of angles at the scale $s$.

\paragraph{PMEAN2}

TBD

Evaluations by Nowak et al.\ \autocite{nowak_sampling_2006} have shown effects of 
