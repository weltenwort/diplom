\chapter{Background \& Related Work}\label{ch:background}
Most approaches can be characterized by looking at three stages in their processing pipeline:

\begin{description}
    \item[Input format] The structure of the input data determines the amount of information available to the subsequent processing steps. Possible preprocessing steps include color space conversion, scaling and edge extraction.
    \item[Extracted features] The large number of coefficients produced by the curvelet transform are reduced to a set of feature coefficients.
    \item[Distance metric] In order to rank the images according to similarity a metric is used to calculate the distance in feature space between two sets of feature coefficients. The selection of a metric is often closely coupled with the feature extraction algorithm.
\end{description}

\section{input format}

\section{features}

\begin{itemize}
    \item bag of features from k-means clustered visual words [video google]
    \item great comparison of sampling for k-means clustered vws [nowak06]
\end{itemize}

\section{metric}

\begin{itemize}
    \item after ranking using euclidean distance, rank by spatial similarity [video google]
    \item Earth Mover's distance? [rubnerljcv00]
\end{itemize}
