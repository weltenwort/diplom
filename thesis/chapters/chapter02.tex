\chapter{Background \& Related Work}\label{ch:background}

Illustration of basic structure of CBIR using local/global features.

Most approaches can be characterized by looking at three stages in their processing pipeline:

\begin{description}
    \item[Input format] The structure of the input data determines the amount of information available to the subsequent processing steps. Possible preprocessing steps include color space conversion, scaling and edge extraction.
    \item[Extracted features] Many algorithms produce a large number of coefficients that can be reduced to a set of feature coefficients using by techniques such as vector quantization or principal component analysis (PCA).
    \item[Distance metric] In order to rank the images according to similarity a metric is used to calculate the distance in feature space between two sets of feature coefficients. The selection of a metric is often closely coupled with the feature extraction algorithm.
\end{description}

\section{input format}
Complete vs incomplete sketches, intra-/cross-domain

\section{features}

\begin{itemize}
    \item bag of features from k-means clustered visual words [video google]
    \item histogram of oriented gradients [chalechale + refs]
\end{itemize}

\section{metric}

\begin{itemize}
    \item after ranking using euclidean distance, rank by spatial similarity [video google]
    \item Earth Mover's distance? [rubnerljcv00]
\end{itemize}
