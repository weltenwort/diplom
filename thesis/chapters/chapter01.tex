\chapter{Introduction}\label{ch:introduction}

\section{Motivation}

Paragraph about increase in visual data, mobile cameras, medicine, etc...

At the core of the research into content-based image retrieval lies the need to
be able to access the growing repositories of visual data in a convenient and
efficient manner.  In this context "convenient" describes the ability for the
user to express the query without a complex reformulation of the intent to make
it accessible to the query processor. At the same time the computational
efficiency becomes more important as the amount of data to search grows. This
issue becomes even more critical as the use of mobile, power-limited devices
increases across many areas of application, such as autonomous vehicles or
handheld augmented reality devices.

Research into text-based information retrieval has brought into existence many
statistical methods to query a potentially large body of text using text as the
query input. This preserves the close mapping of the intent of the user to the
expression of the query and thereby makes the process accessible to users
without knowledge about the internal workings of the retrieval system.
Providing the means to access a large amount of visual data using a system with
similar properties has turned out not to be an easy problem. Using text-based
querying for that purpose depends on the ability to reliably label visual data,
which would require solving the general object recognition problem first
\autocite{smeulders_content-based_2000}. The avoid that obstacle and to free
the retrieval system from the requirement of translating between textual and
visual information, many methods to search an image database using visual
similarity have been developed.

Sematic Gap

\section{Defining Content-Based Image Retrieval}



\begin{itemize}
    \item Definition CBIR
    \item historical intro [CBIR at the end of the early years]
    \item Visual Similarity => Semantic Similarity? Sematic Gap! \autocite{smeulders_content-based_2000}
\end{itemize}


\section{Outline}

Thesis outline goes here
