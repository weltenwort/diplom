\chapter{Introduction}\label{ch:introduction}

\section{Motivation}

Paragraph about increase in visual data, mobile cameras, etc

At the core of the research into content-based image retrieval lies the need to
be able to access the growing repositories of visual data in a convenient and
efficient manner.  In this context "convenient" describes the ability for the
user to express the query without a complex reformulation of the intent to make
it accessible to the query processor. At the same time the computational
efficiency becomes more important as the amount of data to search grows. This
issue becomes even more critical as the use of mobile, power-limited devices
increases across many areas of application, such as autonomous vehicles or
handheld augmented reality devices.

\begin{itemize}
    \item Definition CBIR
    \item historical intro [CBIR at the end of the early years]
    \item Why CBIR: Insufficient Mapping Image <=> Language
        \begin{itemize}
            \item Search by Example
            \item Association Search (Discovery)
        \end{itemize}
    \item Visual Similarity => Semantic Similarity? Sematic Gap! [Smeulders2000]
\end{itemize}


\section{Outline}

Thesis outline goes here
