\begin{tikzpicture}[font=\tiny]
    \matrix[node grid] {
        \node [document node] (dbimg) {$I_{db}$}; &
        \node [operation node] (dbluma) {LUMA}; &
        \node [operation node] (dbcurvelet) {FDCT}; &
        \node [operation node] (dbpmean) {PMEAN}; &&
        \node [operation node] (dbquant) {VQ}; \\
        &&&
        \node [operation node] (cluster) {CLUSTER}; &&&
        \node [operation node] (dist) {HI(B)}; &
        \node [document node] (result) {distances}; \\
        \node [document node] (qimg) {$I_q$}; &
        \node [operation node] (qluma) {LUMA}; &
        \node [operation node] (qcurvelet) {FDCT}; &
        \node [operation node] (qpmean) {PMEAN}; &&
        \node [operation node] (qquant) {VQ}; \\
    };
    { [start chain=going right, every join/.style={connector}]
        \chainin (dbimg);
        \chainin (dbluma) [join];
        \chainin (dbcurvelet) [join];
        \chainin (dbpmean) [join];
        { [start branch]
            \chainin (cluster) [join];
            { [start branch]
                \chainin (qquant) [join=with cluster.-2 by hv connector];
            }
            \chainin (dbquant) [join=with cluster.2 by hv connector];
        }
        \chainin (dbquant) [join];
        \chainin (dist) [join=by hv connector];
    }
    { [start chain=going right, every join/.style={connector}]
        \chainin (qimg);
        \chainin (qluma) [join];
        \chainin (qcurvelet) [join];
        \chainin (qpmean) [join];
        \chainin (qquant) [join];
        \chainin (dist) [join=by hv connector];
        \chainin (result) [join];
    }
\end{tikzpicture}
