\colorlet{lightbg}{yellow!20}
\colorlet{lightborder}{yellow!50!black}
\colorlet{graybg}{black!10}
\colorlet{grayborder}{black!50}

\usetikzlibrary{fit,positioning,chains,scopes,shapes,shapes.multipart,calc}

\tikzset{
	node grid/.style={row sep=1ex, column sep=1.5ex},
	small node/.style={very thick, text height=1ex, text depth=.25ex, rounded corners, node distance=1.5ex},
    split node/.style={rectangle split, rectangle split parts=#1},
	document node/.style={small node, draw=lightborder, fill=lightbg},
	operation node/.style={small node, draw=grayborder, fill=graybg},
	parameter node/.style={small node},
	connector/.style={very thick, draw=grayborder, ->, rounded corners},
	parameter connector/.style={draw=black, ->, rounded corners},
	vh connector/.style={connector, to path={|- (\tikztotarget)}},
	hv connector/.style={connector, to path={-| (\tikztotarget)}},
    hvh connector top/.style={to path={-- ++(1.5ex,0) |- (\tikztotarget.170)}},
    hvh connector bottom/.style={to path={-- ++(1.5ex,0) |- (\tikztotarget.190)}},
    left offset/.style={right=1.5ex}
}

\usepackage{pgfplots}
\usepackage{pgfplotstable}
\usepackage{multirow}
\usepackage{booktabs}


%\pgfplotstableset{col sep=comma}
\pgfplotsset{
    /pgfplots/flexible yticklabels from table/.code n args={2}{%
        \pgfplotstablegetcolumn{#2}\of{#1}\to\pgfplots@yticklabels
        \let\pgfplots@yticklabel=\pgfplots@user@ticklabel@list@y
    },
    hbarplot/.style={
        xbar,
        small,
        y=-\baselineskip,
        enlarge y limits={true, abs value=0.45},
        xmin=0,
        xmax=0.3,
        ytick=\empty,
        xticklabel style={/pgf/number format/.cd,fixed,precision=2},
        nodes near coords,
        nodes near coords align=horizontal,
        every node near coord/.append style={font=\tiny, /pgf/number format/.cd,fixed,precision=3},
        axis x line*=bottom,
        axis y line*=left,
    },
}

\pgfplotstableset{create on use/graph/.style={
    create col/expr=0
    },
    every table/.style={font=\small},
}
\pgfplotstableread[]{results/reference.csv}\resultsreference
\newcommand{\addreferenceplot}{\addplot[const plot, red, update limits=false] table[x=MeanCorrelation, y expr=\coordindex*100-1] {\resultsreference}}

\newcommand{\plotxbars}[1]{%
    \begin{tikzpicture}
        \begin{axis}[
            hbarplot,
            width=6cm,
            ]
            \addplot table[x=MeanCorrelation, y expr=\coordindex] {#1};
            \addreferenceplot;
        \end{axis}
    \end{tikzpicture}%
}

\newcommand{\plottablexbars}[2]{
    \pgfplotstablegetrowsof{#2}
    \let\numberofrows=\pgfplotsretval

    \pgfplotstabletypeset[columns={#1,graph},
      % Booktabs rules
      every head row/.style={after row=\midrule},
      every last row/.style={after row=[3ex]},
      % Set header name
      columns/scales/.style={string type,column name=$N_s$},
      columns/angles/.style={string type,column name=$N_{\theta}$},
      columns/gridsize/.style={string type,column name=$G$},
      columns/patchsize/.style={string type,column name=$P$},
      columns/cannysigma/.style={string type,column name=$\sigma$},
      columns/metric/.style={string type,column name=Metric},
      columns/graph/.style={
        column name={},
        assign cell content/.code={% use \multirow for Z column:
        \ifnum\pgfplotstablerow=0
        \pgfkeyssetvalue{/pgfplots/table/@cell content}
        {\multirow{\numberofrows}{5cm}{\plotxbars{#2}}}%
        \else
        \pgfkeyssetvalue{/pgfplots/table/@cell content}{}%
        \fi
        }
      },
    ]{#2}
}
